
Искомую оценку элемента $x_{11}$ оператора $X$, являющегося решением нелинейного уравнения (\ref{eq:x_main}) мы получим, получив оценку $\norm{X_{11}}$. Для оценки $\norm{X_{11}}$ в свою очередь требуется оценка $\norm{X_{21}}$ и разрешимость уравнения (\ref{eq:x21}). Потому сформулируем и докажем следующую теорему:

\noindent\textbf{Теорема~2.}
{ \it Пусть выполнено неравенство
\begin{align}
d = 2\gamma^2 (b_{12}b_{21})^{\frac{1}{2}} < 1, \label{coef}
\end{align}
а также выполняются условия предыдущей теоремы. 
Тогда для нелинейных уравнений \eqref{eq:x11} - \eqref{eq:x21} имеют место следующие оценки:
\begin{align*}
&\norm{X_{11} - B_{11}} \leq \frac{2\gamma b_{12}b_{21}}{1 + (1 - 4\gamma^2 b_{21}b_{12})^{\frac{1}{2}}}
&\norm{X_{21}} \leq \frac{2b_{21}}{1 + (1 - 4\gamma^2 b_{21}b_{12})^{\frac{1}{2}}},
\end{align*}
где $\norm{B_{ij}} = b_{ij},\ i,j = 1,2.$}

\noindent\textbf{Доказательство.}

Рассмотрим оператор $\Phi_1(X_{21}),$ определяемый уравнением \eqref{eq:x21}. Найдем шар $B(r') = \{ X \in  \mathfrak{X}_{21}: \norm{X} < r' \norm{B_{21}} \}$ из пространства $ \mathfrak{X}_{21},$ который оператор $\Phi_1(X_{21})$ переводит в себя, т.\! е. $\norm{\Phi_1(X_{21})} \leq r' \norm{B_{21}}$ для любого $X \in B(r').$ Обозначим $r' = r + 1.$
\begin{align*}
&\norm{\Phi_1(X_{21})} = - (\Gamma X_{21})B_{12}\Gamma X_{21} + B_{21} \leq \\
&\leq \gamma^2 b_{12}\norm{X_{21}}^2 + b_{21} \leq \\ 
&\leq \gamma^2 b_{21}^2 b_{12}r'^2 + b_{21} \leq b_{21}r'
\end{align*}
Получаем неравенство:
$$
\gamma^2 b_{21}^2 b_{12}r'^2 - r' + 1 \leq 0.
$$
Покажем что квадратное уравнение относительно $r'$
$$
\gamma^2 b_{21}^2 b_{12}r'^2 - r' + 1 = 0
$$
имеет хотя бы один действительный положительный корень. Воспользуемся тем, что $\gamma\norm{B} < \frac{1}{2}$ и $b_{ij} \leq \norm{B}.$
\begin{align*}
&D = 1 - 4\gamma^2 b_{21}b_{12}; \\
&1 - 4\gamma^2 b_{21}b_{12} > 1 - 4\cdot\frac{1}{4} = 0
\end{align*}
Получаем, что 
$$
	r_{1,2}' = \frac{1 \pm (1 - 4\gamma^2 b_{21}b_{12})^{\frac{1}{2}}}{2\gamma^2 b_{21}b_{12}}
$$	
Рассмотрим решение $r' = \frac{1 - (1 - 4\gamma^2 b_{21}b_{12})^{\frac{1}{2}}}{2\gamma^2 b_{21}b_{12}},$ докажем, что оно больше нуля.
\begin{align*}
\frac{1 - (1 - 4\gamma^2 b_{21}b_{12})^{\frac{1}{2}}}{2\gamma^2 b_{21}b_{12}} = \frac{2}{1 + (1 - 4\gamma^2 b_{21}b_{12})^{\frac{1}{2}}} > 0
\end{align*}
Получаем, что в качестве радиуса шара можно взять число
$$
 r' b_{21} = \frac{2b_{21}}{1 + (1 - 4\gamma^2 b_{21}b_{12})^{\frac{1}{2}}}.
$$
Для любой пары операторов $Y_1, Y_2$ из шара $B(r')$ имеет место оценка
\begin{align*}
&\norm{\Phi_1(Y_1) - \Phi_1(Y_2)} = \|- (\Gamma Y_1)B_{12}\Gamma Y_1 + B_{21} + \\
& + (\Gamma Y_2)B_{12}\Gamma Y_2 - B_{21}\| \leq (\gamma^2 b_{12}(\norm{Y_1} + \\
& + \norm{Y_2}))\norm{Y_1 - Y_2} \leq (\frac{2\gamma^2 (b_{12}b_{21})^{\frac{1}{2}}}{1 + (1 - 4\gamma^2 b_{21}b_{12})^{\frac{1}{2}}})\norm{Y_1 - Y_2} \leq d\norm{Y_1 - Y_2}.
\end{align*}
Из условия \eqref{coef} следует, что оператор $\Phi_1$ является оператором сжатия в шаре $B(r').$ Тогда уравнение \eqref{eq:x21} имеет единственное решение $X_{21}$ в этом шаре, которое можно найти методом простых итераций. Так как $X_{21}$ принадлежит шару $B(r'),$ то справедливо неравенство
\begin{align*}
\norm{X_{11} - B_{11}} = \norm{B_{12}\Gamma X_{21}} \leq \gamma b_{12} \norm{X_{21}} \leq \frac{2\gamma b_{12}b_{21}}{1 + (1 - 4\gamma^2 b_{21}b_{12})^{\frac{1}{2}}}.
\end{align*}
Теорема доказана.

Таким образом мы имеем следующую оценку для $\norm{X_{11}}$
$$
	\norm{X_{11}} \leq \frac{2\gamma b_{12}b_{21}}{1 + (1 - 4\gamma^2 b_{21}b_{12})^{\frac{1}{2}}} \leq \frac{1}{2\gamma} - \frac{1 - (1 - 4\gamma^2 b_{21}b_{12})^{\frac{1}{2}}}{4\gamma^2 b_{21}b_{12}},
$$
где $\gamma = \norm{(\alpha I - B)^{-1}}, b_{ij} = \norm{B_{ij}}, i,j=1,2$ 