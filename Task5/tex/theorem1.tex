
Будем искать такой оператор $X_0 \in \mathcal{U}$, чтобы выполнялось равенство
\begin{equation}\label{eq:som_eq}
	(\mathcal{A}-\mathcal{B})(I + \Gamma X_0) = (I + \Gamma X_0)(\mathcal{A}-\mathcal{J}X_0).
\end{equation}
При условии $\norm{\Gamma X_0} < 1$ (тогда оператор $I + \Gamma X_0$ обратим) равенство (\ref{eq:som_eq}) означает подобие операторов $\mathcal{A}-\mathcal{B}$ и $\mathcal{A}-\mathcal{J}X_0$. Таким образом задача оценки первого собственного значения возмущенного оператора $\mathcal{A}-\mathcal{B}$ сводится к задаче оценки собственного значения оператора $\mathcal{A}-\mathcal{J}X_0$, равного $a_{11} - x^{0}_{11}$ в силу указанного разложения пространства $\mathfrak{X}$. Необходимо получить оценку $x^{0}_{11}$ матрицы оператора $X_0$. Для этого сначала преобразуем равенство (\ref{eq:som_eq}):
\begin{gather}
	\mathcal{A} + \mathcal{A} \Gamma X - \mathcal{B} - \mathcal{B} \Gamma X = \mathcal{A} - \mathcal{J} X + \Gamma X \mathcal{A} - \Gamma X \mathcal{J} X; \notag \\
	\mathcal{A} \Gamma X - \Gamma X \mathcal{A} - \mathcal{B} \Gamma X + \mathcal{J} X + \Gamma X \mathcal{J} X - \mathcal{B} = 0; \notag \\
	X - \mathcal{J} X - \mathcal{B} \Gamma X + \mathcal{J} X + \Gamma X \mathcal{J} X - \mathcal{B} = 0;  \notag \\
	X = \mathcal{B} \Gamma X - \Gamma X \mathcal{J} X + \mathcal{B}. \label{eq:x_first}
\end{gather}
Применим к уравнению (\ref{eq:x_first}) трансформатор $\mathcal{J}$:
$$
	\mathcal{J}X = \mathcal{J}\mathcal{B}\Gamma X + \mathcal{J}\mathcal{B},
$$ 
и подставим $\mathcal{J}X$ в уравнение (\ref{eq:x_first}). Учитывая, что $\mathcal{JB} = 0 $ получаем нелинейное уравнение
\begin{equation}\label{eq:x_main}
	X = \mathcal{B}\Gamma X - \Gamma X \mathcal{J}(\mathcal{B}\Gamma X) - \mathcal{B} = \Phi(X).
\end{equation}

Пусть $P_1, P_2$ -- проекторы относительно разложения пространства $\mathfrak{X} = \mathfrak{X}_1 \oplus \mathfrak{X}_2$. 
Заметим, что $\forall X \in \spaceend\mathfrak{X}$ выполняются следующие два равенства:
\begin{enumerate}
	\item $\mathcal{J}X = P_1 X P_1 + P_2 X P_2;$
	\item $P_i(\Gamma X)P_j = \Gamma(P_i X P_j),\ i,j = 1,2$, и $P_i(\Gamma X)P_i = 0,\ i = 1,2$.
\end{enumerate}
Применим операторы $P_1 \text{ и } P_2$ к обеим частям уравнения \eqref{eq:x_main} и воспользуемся равенствами, приведенными выше.
\begin{enumerate}
	\item Применим справа и слева проектор $P_1$:
	\begin{align}
		\begin{split}
			P_1 X P_1 &= P_1 \mathcal{B}\Gamma X P_1 - P_1\Gamma X \mathcal{J}(\mathcal{B}\Gamma X)P_1 - P_1 \mathcal{B} P_1 = \\
			&= P_1\mathcal{B}(P_1 + P_2)\Gamma X P_1 - \Gamma P_1 X (P_1 \mathcal{B}\Gamma X P_1 + P_2 \mathcal{B}\Gamma X P_2) P_1 - 0 = \\
			&= 0 + \mathcal{B}_{12}\Gamma X_{21} - \Gamma P_1 X P_1 \mathcal{B}_{12} \Gamma X_{21} P_1 = 
			\mathcal{B}_{12}\Gamma X_{21}; \notag
		\end{split} \\
		X_{11} &= \mathcal{B}_{12}\Gamma X_{21}, \label{eq:x11}
		\intertext{где $X_{11}$ -- сужение оператора $P_1 X P_1$ на пространство $\mathfrak{X}_1$ со значениями в $\mathfrak{X}_1$.} \notag
	\end{align}
	
	\item Применим справа проектор $P_1$, а слева $P_2$:
	\begin{align}
		P_2 X P_1 &= P_2 \mathcal{B}\Gamma X P_1 - P_2\Gamma X \mathcal{J}(\mathcal{B}\Gamma X)P_1 - P_2 \mathcal{B} P_1; \notag \\
		X_{21} &= - (\Gamma X_{21})\mathcal{B}_{12}\Gamma X_{21} + \mathcal{B}_{21} = \Phi_1(X_{21}); \label{eq:x21}
	\end{align}

	\item Применим справа и слева проектор $P_2$:
	\begin{align}
		P_2 X P_2 &= P_2 \mathcal{B}\Gamma X P_2 - P_2\Gamma X \mathcal{J}(\mathcal{B}\Gamma X)P_2 - P_2 \mathcal{B} P_2; \notag \\
		X_{22} &= \mathcal{B}_{21}\Gamma X_{12};\label{eq:x22}
	\end{align}

	\item Применим справа проектор $P_2$, а слева $P_1$:
	\begin{align}
		P_1 X P_2 &= P_1 \mathcal{B}\Gamma X P_2 - P_1\Gamma X \mathcal{J}(\mathcal{B}\Gamma X)P_2 - P_1 \mathcal{B} P_2; \notag \\
		X_{12} &= - (\Gamma X_{12})\mathcal{B}_{21}\Gamma X_{12} + \mathcal{B}_{12} = \Phi_2(X_{12}).\label{eq:x12}
	\end{align}
	
\end{enumerate}
Получили четыре уравнения, причем уравнения \eqref{eq:x12} и \eqref{eq:x21} независимы от остальных уравнений. Разрешимость уравнения \eqref{eq:x_main} будет доказана если, будет доказана разрешимость уравнений \eqref{eq:x12} и \eqref{eq:x21}.  

\noindent\textbf{Теорема~1.}
{ \it Пусть выполнено неравенство
$$
\gamma\norm{\mathcal{B}} < \frac{1}{2},
$$
где $\gamma = \norm{(\alpha I - B)^{-1}},$ тогда нелинейные уравнения \eqref{eq:x12} и \eqref{eq:x21} имеют единственные решения $X_{12}, X_{21}$ в шаре с центром в точке $\mathcal{B}$ и радиусом $\norm{\mathcal{B}}$. $X_{12}$ и $X_{21}$ можно найти методом простых итераций, если в качестве первого приближения взять $X_{ij} = 0.$}

\noindent\textbf{Доказательство.}
\begin{enumerate}
\item Рассмотрим уравнение $\Phi_1.$ Найдем такой шар с центром в точке $\mathcal{B}$, который отображение $\Phi_1$ переводит сам в себя, т.е. если $\norm{X_{21} - \mathcal{B}} < r\norm{\mathcal{B}} (\text{или} \norm{X_{21}} < (r + 1)\norm{\mathcal{B}})$, то и $\norm{\Phi_1(X_{21}) - \mathcal{B}} < r\norm{\mathcal{B}}$.
\begin{align*}
\|\Phi_1(X_{21}) - \mathcal{B}\| &\leq \|-(\Gamma X_{21})\mathcal{B}_{12}\Gamma X_{21}\| \leq \gamma^2 \|\mathcal{B}\|~ \|X_{21}\|^2 \leq \\ 
&\leq \gamma^2 \|\mathcal{B}\|^3 (r+1)^2 \leq r\|\mathcal{B}\|.
\end{align*}
Символом $r'$ обозначим $r+1$. Получили квадратное уравнение относительно $r'$:
$$
r'^2 \gamma^2 \|\mathcal{B}\|^2 - r' +1 \leq 0.
$$
Пусть $\varepsilon=\gamma \|\mathcal{B}\|$, тогда:
\begin{align*}
&\varepsilon^2 r'^2 - r' + 1 \leq 0, \\  
&\mathcal{D}= 1 - 4\varepsilon^2,
\end{align*}
для существования корней необходимо выполнение условия $\mathcal{D}~>~0$.
Получаем квадратное неравенство относительно $\varepsilon.$
\begin{align*}
&1 - 4\varepsilon^2 > 0 \\
& \varepsilon = \frac{1}{2}, \varepsilon = - \frac{1}{2}. 
\end{align*}
Получаем, что $\varepsilon \in (0; \frac{1}{2}),$ т.\! е. $\varepsilon < \frac{1}{2}$
Найдем $r'$ при условии, что $\varepsilon = \frac{1}{2}:$
$$
r' = \frac{1 \pm \sqrt{1 - 4\varepsilon^2}}{2\varepsilon^2} = 2.
$$
Тогда получаем, что $r = 1.$

Проверим сжимаемость отображения $\Phi_1$ в этом шаре:
\begin{align*}
\|\Phi_1(X_{21}) - \Phi_1(Y_{21}) \| &= \| -(\Gamma X_{21})\mathcal{B}_{12}\Gamma X_{21} + (\Gamma Y_{21})\mathcal{B}_{12}\Gamma Y_{21}\|  = \\ 
&= \| -(\Gamma X_{21})\mathcal{B}_{12}\Gamma X_{21} + (\Gamma Y_{21})\mathcal{B}_{12}\Gamma Y_{21} - \\
& \phantom{=} -(\Gamma Y_{21})\mathcal{B}_{12}\Gamma X_{21} + (\Gamma Y_{21})\mathcal{B}_{12}\Gamma X_{21}\| = \\
 &= \| - (\Gamma (X_{21} - Y_{21})) \mathcal{J}(\mathcal{B}_{12}\Gamma X_{21}) - \\
 & \phantom{=} - (\Gamma Y_{21})\mathcal{J}(\mathcal{B}_{12}\Gamma (X_{21} - Y_{21})) \| \leqslant \\
 &\leqslant 2\varepsilon ^2 (r + 1)\| X_{21} - Y_{21} \| 
\end{align*}
Возьмем $r = 1.$ Покажем, что $2\varepsilon ^2 (r + 1) < 1.$
$$
4\varepsilon ^2 < 4 \cdot \frac{1}{4} = 1.
$$

Тогда получаем, что отображение $\Phi_1$ переводит шар с центром в точке $\mathcal{B}$ и радиусом $2\|\mathcal{B}\|$ в себя и является на этом шаре сжимающим отображением, следовательно, существует внутри шара неподвижная точка отображения $\Phi_1$ , являющаяся единственным решением уравнения \eqref{eq:x_main} и ее можно найти по методу простых итераций, используя в качестве первого приближения нулевой оператор.

\item Проведем аналогичные рассуждения для уравнения $\Phi_2.$ Найдем такой шар с центром в точке $\mathcal{B}$, который отображение $\Phi_2$ переводит сам в себя, т.е. если $\norm{X_{12} - \mathcal{B}} < r\norm{\mathcal{B}} (\text{или} \norm{X_{12}} < (r + 1)\norm{\mathcal{B}})$, то и $\norm{\Phi_2(X_{12}) - \mathcal{B}} < r\norm{\mathcal{B}}$.
\begin{align*}
\|\Phi_2(X_{12}) - \mathcal{B}\| &\leq \|-(\Gamma X_{12})\mathcal{B}_{21}\Gamma X_{12}\| \leq \gamma^2 \|\mathcal{B}\|~ \|X_{12}\|^2 \leq \\ 
&\leq \gamma^2 \|\mathcal{B}\|^3 (r+1)^2 \leq r\|\mathcal{B}\|.
\end{align*}
Символом $r'$ обозначим $r+1$. Получили квадратное уравнение относительно $r'$:
$$
r'^2 \gamma^2 \|\mathcal{B}\|^2 - r' +1 \leq 0.
$$
Пусть $\varepsilon=\gamma \|\mathcal{B}\|$, тогда:
\begin{align*}
&\varepsilon^2 r'^2 - r' + 1 \leq 0, \\  
&\mathcal{D}= 1 - 4\varepsilon^2,
\end{align*}
для существования корней необходимо выполнение условия $\mathcal{D}~>~0$.
Получаем квадратное неравенство относительно $\varepsilon.$
\begin{align*}
&1 - 4\varepsilon^2 > 0 \\
& \varepsilon = \frac{1}{2}, \varepsilon = - \frac{1}{2}. 
\end{align*}
Получаем, что $\varepsilon \in (0; \frac{1}{2}),$ т.\! е. $\varepsilon < \frac{1}{2}$
Найдем $r'$ при условии, что $\varepsilon = \frac{1}{2}:$
$$
r' = \frac{1 \pm \sqrt{1 - 4\varepsilon^2}}{2\varepsilon^2} = 2.
$$
Тогда получаем, что $r = 1.$

Проверим сжимаемость отображения $\Phi_2$ в этом шаре:
\begin{align*}
\|\Phi_2(X_{12}) - \Phi_2(Y_{12}) \| &= \| -(\Gamma X_{12})\mathcal{B}_{21}\Gamma X_{12} + (\Gamma Y_{12})\mathcal{B}_{21}\Gamma Y_{12}\|  = \\ 
&= \| -(\Gamma X_{12})\mathcal{B}_{21}\Gamma X_{12} + (\Gamma Y_{12})\mathcal{B}_{21}\Gamma Y_{12} - \\
& \phantom{=} -(\Gamma Y_{12})\mathcal{B}_{21}\Gamma X_{12} + (\Gamma Y_{12})\mathcal{B}_{21}\Gamma X_{12}\| = \\
 &= \| - (\Gamma (X_{12} - Y_{12})) \mathcal{J}(\mathcal{B}_{21}\Gamma X_{12}) - \\
 & \phantom{=} - (\Gamma Y_{12})\mathcal{J}(\mathcal{B}_{21}\Gamma (X_{12} - Y_{12})) \| \leqslant \\
 &\leqslant 2\varepsilon ^2 (r + 1)\| X_{12} - Y_{12} \| 
\end{align*}
Возьмем $r = 1.$ Покажем, что $2\varepsilon ^2 (r + 1) < 1.$
$$
4\varepsilon ^2 < 4 \cdot \frac{1}{4} = 1.
$$

Тогда получаем, что отображение $\Phi_2$ переводит шар с центром в точке $\mathcal{B}$ и радиусом $2\|\mathcal{B}\|$ в себя и является на этом шаре сжимающим отображением, следовательно, существует внутри шара неподвижная точка отображения $\Phi_2$ , являющаяся единственным решением уравнения \eqref{eq:x_main} и ее можно найти по методу простых итераций, используя в качестве первого приближения нулевой оператор.   
\end{enumerate}
Теорема доказана.
\hfill

Найдя $X_{21}$ и $X_{12},$ подставим их в уравнения \eqref{eq:x11} и \eqref{eq:x22} соответственно, получим $X_{11}$ и $X_{22}.$ Отсюда следует, что уравнение \eqref{eq:x_main} разрешимо, и его решение находиться по формуле:
$$
	X_0 = \begin{pmatrix}
		X_{11} & X_{12} \\
		X_{21} & X_{22}
	\end{pmatrix}.
$$ 
