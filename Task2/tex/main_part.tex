
Будем искать такой оператор $X_0 \in \spaceend\field{C}^n$, чтобы выполнялось равенство
\begin{equation}\label{eq:som_eq}
	(A-B)(I + \Gamma X_0) = (I + \Gamma X_0)(A-\mathcal{J}X_0).
\end{equation}
При условии $\norm{\Gamma X_0} \leqslant 1$ (тогда оператор $I + \Gamma X_0$ обратим) равенство (\ref{eq:som_eq}) означает подобие операторов $A-B$ и $A-\mathcal{J} X_0$. Таким образом задача оценки первого собственного значения возмущенного оператора $A-B$ сводится к задаче оценки первого собственного значения оператора $A-\mathcal{J}X_0$, равного $a_{11} - x^{0}_{11}$ в силу указанного разложения пространства $\field{C}^n$. Необходимо получить оценку $x^{0}_{11}$ матрицы оператора $X_0$. Для этого сначала преобразуем равенство (\ref{eq:som_eq}):
\begin{gather}
	A + A\Gamma X - B - B\Gamma X = A - \mathcal{J}X + \Gamma X A - \Gamma X \mathcal{J}X; \notag \\
	A\Gamma X - \Gamma X A - B\Gamma X + \mathcal{J}X + \Gamma X \mathcal{J}X - B = 0; \notag \\
	X - \mathcal{J}X - B\Gamma X + \mathcal{J}X + \Gamma X \mathcal{J}X - B = 0;  \notag \\
	X = B\Gamma X - \Gamma X \mathcal{J}X + B. \label{eq:x_first}
\end{gather}
Применим к уравнению (\ref{eq:x_first}) трансформатор $\mathcal{J}$:
$$
	\mathcal{J}X = \mathcal{J}B\Gamma X + \mathcal{J}B,
$$ 
и подставим $\mathcal{J}X$ в уравнение (\ref{eq:x_first}). Получаем нелинейное уравнение
\begin{equation}\label{eq:x_main}
	X = B\Gamma X - \Gamma X \mathcal{J}(B\Gamma X) - \Gamma X \mathcal{J}B - B = \Phi(X).
\end{equation}
\noindent\textbf{Теорема~1.}
{ \it Пусть выполнено неравенство
$$
\gamma\norm{B} < \frac{1}{4},
$$
тогда нелинейное уравнение \eqref{eq:x_main} имеет единственное решение $X_0$ в шаре с центром в нуле и радиусом $4\norm{B}$, на котором достигается равенство \eqref{eq:som_eq}. $X_0$ можно найти методом простых итераций, если в качестве первого приближения взять $X = 0.$}

\noindent\textbf{Доказательство.}
Найдем такой шар с центром в точке 0, который отображение $\Phi$ переводит сам в себя, т.е. если $\|X\|<r\|B\|$, то и $\|\Phi(X)\|<r\|B\|$.
\begin{align*}
\|\Phi(X)\| &\leq \|B(\Gamma X)-(\Gamma X)(\mathcal{J}B)-(\Gamma X)\mathcal{J}(B\Gamma X)+B\| \leq \\
&\leq \gamma \|B\|~ \|X\|+\gamma \|B\|~ \|X\|+\gamma^2 \|B\|~ \|X\|^2 + \|B\| \leq \\ 
&\leq \gamma r \|B\|^2+\gamma r \|B\|^2+\gamma^2 \|B\|^3 r^2 + \|B\| \leq r\|B\|.
\end{align*}
Получили квадратное уравнение относительно $r$:
$$
r^2 \gamma^2 \|B\|^2 + r (2\gamma \|B\|-1)+1 \leq 0.
$$
Пусть $\varepsilon=\gamma \|B\|$, тогда:
\begin{align*}
&\varepsilon^2 r^2 + (2 \varepsilon - 1)r + 1 \leq 0, \\  
&\mathcal{D}=(2\varepsilon-1)^2 - 4\varepsilon^2 = 4\varepsilon^2-4\varepsilon+1-4\varepsilon^2=1-4\varepsilon,
\end{align*}
для существования корней необходимо выполнение условия $\mathcal{D}~>~0$, откуда $1~-~4\varepsilon~>~0$, т.е. $\varepsilon~<~\frac{1}{4}$.

Найдем оценку на радиус шара r:
$$
r=\frac{1-2\varepsilon\mp\sqrt{1-4\varepsilon}}{2\varepsilon^2};
$$
при $\varepsilon=\frac{1}{4}, r=4.$
Проверим сжимаемость отображения $\Phi$ в этом шаре:
\begin{align*}
\|\Phi (X) &- \Phi (Y) \| = \| B\Gamma X - \Gamma X\mathcal{J}B - \Gamma X\mathcal{J}(B\Gamma X) + B - B\Gamma Y + \\ 
 & \phantom{=} + \Gamma Y\mathcal{J}B + \Gamma Y\mathcal{J}(B\Gamma Y) -B\| = \| B\Gamma (X-Y) - \Gamma (X-Y) \mathcal{J}B - \\
 & \phantom{=} - \Gamma X\mathcal{J}(B\Gamma X) + \Gamma Y\mathcal{J} (B\Gamma Y) - \Gamma Y\mathcal{J}(B\Gamma X) + \Gamma Y\mathcal{J}(B\Gamma X) \| = \\
 &= \| B\Gamma(X - Y) - \Gamma (X-Y)\mathcal{J}B - \Gamma (X- Y) \mathcal{J}(B\Gamma X) - \\
 & \phantom{=} - \Gamma Y\mathcal{J}(B\Gamma (X - Y)) \| \leqslant 2\varepsilon \| X-Y \| + 2\varepsilon ^2 r\| X-Y \| = \\
 & = (2\varepsilon + 2\varepsilon ^2 r)\| X-Y \| 
\end{align*}
Возьмем $r = 4.$ Покажем, что $2\varepsilon + 2\varepsilon ^2 r < 1.$
$$
2\varepsilon + 2\varepsilon ^2 4 < 2 \cdot \frac{1}{4} + 2 \cdot \frac{1}{16} 4 = \frac{1}{2} + \frac{1}{8} 4 = 1.
$$

Тогда получаем, что отображение отображение $\Phi$ переводит шар с центром в нуле и радиусом $4\|B\|$ в себя и является на этом шаре сжимающим отображением, следовательно, существует внутри шара неподвижная точка отображения $\Phi$ , являющаяся единственным решением уравнения \eqref{eq:x_main} и ее можно найти по методу простых итераций, используя в качестве нулевого приближения нулевой оператор. Теорема доказана.
\hfill

Пусть $P_1,P_2$ --- проекторы, ассоциированные с указанным в начале разложением пространства $\field{C}^n$. Заметим, что $\forall X \in \spaceend\field{C}^n$ выполняется:
\begin{enumerate}
	\item $\mathcal{J}X = P_1 X P_1 + P_2 X P_2;$
	\item $P_i(\Gamma X)P_j = \Gamma(P_i X P_j),\ i,j = 1,2$, и $P_i(\Gamma X)P_i = 0,\ i = 1,2$.
\end{enumerate}
Таким образом пространство $\field{C}^n$ можно представить в виде прямой суммы $\mathfrak{X} = \mathfrak{X}_{11} \oplus \mathfrak{X}_{12} \oplus \mathfrak{X}_{21} \oplus \mathfrak{X}_{22}$ подпространств, где $\mathfrak{X}_{ij} = \{ P_i \mathfrak{X} P_j, X \in \spaceend\field{C}^n \},\\ \ i,j = 1,2$.
Через $X_{ij}$ будем обозначать оператор $P_i X P_j,\ i,j=1,2$, таким образом $X = (P_1 + P_2)X(P_1 + P_2) = X_{11} + X_{12} + X_{21} + X_{22},\ X \in \spaceend\field{C}^n$.
Применим операторы $P_1 \text{ и } P_2$ к обеим частям уравнения (\ref{eq:x_main}).
\begin{enumerate}
	\item Применим справа и слева проектор $P_1$:
	\begin{align}
		P_1 X P_1 &= P_1 B\Gamma X P_1 - P_1\Gamma X \mathcal{J}(B\Gamma X)P_1 - P_1\Gamma X \mathcal{J}B P_1 - P_1 B P_1; \notag \\
		\begin{split}
			X_{11} &= (B_{11} + B_{12})(\Gamma X_{11} + \Gamma X_{21}) - \\
			&- (\Gamma X_{11} + \Gamma X_{12})\mathcal{J}((B_{11} + B_{12})(\Gamma X_{11} + \Gamma X_{21})) - \\
			&- (\Gamma X_{11} + \Gamma X_{12})(\mathcal{J}B_{11} + \mathcal{J}B_{21}) + B_{11}; \notag
		\end{split}
		\intertext{Будем учитывать, что $\mathcal{J}X_{12} = \mathcal{J}X_{21} = 0$ и $\Gamma X_{11} = \Gamma X_{22} = 0$.}
		\begin{split}
			X_{11} &= (B_{11} + B_{12})\Gamma X_{21} - \Gamma X_{12}\mathcal{J}((B_{11} + B_{12})\Gamma X_{21}) - \\
			&- \Gamma X_{12}\mathcal{J}B_{11} + B_{11}; \notag
		\end{split} \\
		X_{11} &= B_{12}\Gamma X_{21}  + B_{11};\label{eq:x11}
	\end{align}

	\item Применим справа проектор $P_1$, а слева $P_2$:
	\begin{align}
		P_2 X P_1 &= P_2 B\Gamma X P_1 - P_2\Gamma X \mathcal{J}(B\Gamma X)P_1 - P_2\Gamma X \mathcal{J}B P_1 - P_2 B P_1; \notag \\
		X_{21} &= (B_{11} + B_{22})\Gamma X_{21} - \Gamma X_{21}\mathcal{J}(B_{12}\Gamma X_{21}) - \Gamma X_{21}B_{11} + B_{21}; \notag \\
		X_{21} &= B_{22}\Gamma X_{21} - (\Gamma X_{21})B_{12}\Gamma X_{21} - (\Gamma X_{21})B_{11} + B_{21}; \label{eq:x21}
	\end{align}

\end{enumerate}

Искомую оценку элемента $x_{11}$ оператора $X$, являющегося решением нелинейного уравнения (\ref{eq:x_main}) мы получим, получив оценку $\norm{X_{11}}$. Для оценки $\norm{X_{11}}$ в свою очередь требуется оценка $\norm{X_{21}}$ и разрешимость уравнения (\ref{eq:x21}). Потому сформулируем и докажем следующую теорему:

\noindent\textbf{Теорема~2.}
{ \it Пусть выполнено неравенство
\begin{align}
d = \gamma b_{22} + \gamma b_{11} + 2\gamma^2 (b_{12}b_{21})^{\frac{1}{2}} < 1. \label{coef}
\end{align}
Тогда нелинейное уравнение \eqref{eq:x21} имеет единственное решение, которое можно найти методом простых итераций, и имеют место следующие оценки:
\begin{align*}
&\norm{X_{11} - B_{11}} \leq \frac{2\gamma b_{12}b_{21}}{1 - \gamma b_{22} - \gamma b_{11} + q}
&\norm{X_{21}} \leq \frac{2b_{21}}{1 - \gamma b_{22} - \gamma b_{11} + q},
\end{align*}
где $q = ((\gamma b_{22} + \gamma b_{11} - 1)^2 - 4\gamma b_{21}b_{12})^{\frac{1}{2}}.$}

\noindent\textbf{Доказательство.}

Рассмотрим оператор $\Phi_1(X_{21}),$ определяемый уравнением \ref{eq:x21}. Найдем шар $B(r_1) = \{ X \in  \mathfrak{X}_{21}: \norm{X} < r_1 \}$ из пространства $ \mathfrak{X}_{21},$ который оператор $\Phi_1(X_{21})$ переводит в себя, т.\! е. $\norm{\Phi_1(X_{21})} \leq r_1$ для любого $X \in B(r_1).$ Обозначим $r_1 = rb_{21}.$
\begin{align*}
&\norm{\Phi_1(X_{21})} = \norm{B_{22}\Gamma X_{21} - (\Gamma X_{21})B_{12}\Gamma X_{21} - (\Gamma X_{21})B_{11} + B_{21}} \leq \\
&\leq b_{22}\gamma\norm{X_{21}} + \gamma b_{11}\norm{X_{21}} + \gamma^2 b_{12}\norm{X_{21}}^2 + b_{21} \leq \\ 
&\leq \gamma b_{22}b_{21}r + \gamma b_{11}b_{21}r + \gamma^2 b_{21}^2 b_{12}r^2 + b_{21} \leq b_{21}r
\end{align*}
Получаем неравенство:
$$
\gamma^2 b_{21}^2 b_{12}r^2 + (\gamma b_{22} + \gamma b_{11} - 1)r + 1 \leq 0.
$$
Покажем что квадратное уравнение относительно $r$
$$
\gamma^2 b_{21}^2 b_{12}r^2 + (\gamma b_{22} + \gamma b_{11} - 1)r + 1 = 0
$$
имеет хотя бы один действительный положительный корень. Воспользуемся тем, что $\gamma\norm{B} < \frac{1}{4}$ и $b_{ij} \leq \norm{B}.$
\begin{align*}
&D = (\gamma b_{22} + \gamma b_{11} - 1)^2 - 4\gamma b_{21}b_{12} \\
&(\gamma b_{22} + \gamma b_{11} - 1)^2 - 4\gamma^2 b_{21}b_{12} > (1 - \frac{1}{4} - \frac{1}{4}) - 4\cdot\frac{1}{16} = 0
\end{align*}
Получаем, что 
$$
	r_{1,2} = \frac{1 - \gamma b_{22} - \gamma b_{11} \pm q}{2\gamma^2 b_{21}b_{12}}
$$
Рассмотрим решение $r = \frac{1 - \gamma b_{22} - \gamma b_{11} - q}{2\gamma^2 b_{21}b_{12}},$ докажем, что оно больше нуля.
\begin{align*}
\frac{1 - \gamma b_{22} - \gamma b_{11} - q}{2\gamma^2 b_{21}b_{12}} = \frac{2}{1 - \gamma b_{22} - \gamma b_{11} + q} > 0
\end{align*}
Получаем, что в качестве радиуса шара можно взять число
$$
 rb_{21} = \frac{2b_{21}}{1 - \gamma b_{22} - \gamma b_{11} + q}.
$$
Для любой пары операторов $Y_1, Y_2$ из шара $B(r_1)$ имеет место оценка
\begin{align*}
&\norm{\Phi_1(Y_1) - \Phi_1(Y_2)} = \|B_{22}\Gamma Y_1 - (\Gamma Y_1)B_{12}\Gamma Y_1 - (\Gamma Y_1)B_{11} + B_{21} - \\
& - B_{22}\Gamma Y_2 + (\Gamma Y_2)B_{12}\Gamma Y_2 + (\Gamma Y_2)B_{11} - B_{21}\| \leq (\gamma b_{22} + \gamma b_{11} + \\
& + \gamma^2 b_{12}(\norm{Y_1} + \norm{Y_2}))\norm{Y_1 - Y_2} \leq (\gamma b_{22} + \gamma b_{11} + \\ & + \frac{2\gamma^2 (b_{12}b_{21})^{\frac{1}{2}}(1 - \gamma b_{22} - \gamma b_{11})}{1 - \gamma b_{22} - \gamma b_{11} + q})\norm{Y_1 - Y_2} \leq d\norm{Y_1 - Y_2}.
\end{align*}
Из условия \eqref{coef} следует, что оператор $\Phi_1$ является оператором сжатия в шаре $B(r_1).$ Тогда уравнение \eqref{eq:x21} имеет единственное решение $X_{21}$ в этом шаре, которое можно найти методом простых итераций. Так как $X_{21}$ принадлежит шару $B(r_1),$ то справедливо неравенство
\begin{align*}
\norm{X_{11} - B_{11}} = \norm{B_{12}\Gamma X_{21}} \leq \gamma b_{12} \norm{X_{21}} \leq \frac{2\gamma b_{12}b_{21}}{1 - \gamma b_{22} - \gamma b_{11} + q}.
\end{align*}
Теорема доказана.
\hfill

