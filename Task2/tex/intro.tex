Дан линейный оператор $\mathbb{A} \in \spaceend\field{C}^n$ со своей матрицей $\mathscr{A} = \left( a_{ij} \right)$, внедиагональные элементы которой малы по сравнению с диагональными. Представим оператор $\mathbb{A}$ в виде разности $\mathbb{A} = A-B$ двух линейных операторов $A, B \in \spaceend\field{C}^n$, заданных матрицами $\mathcal{A} \text{ и } \mathcal{B}$ соответственно:

$$
	\mathcal{A} = \begin{pmatrix}
		a_{11} & 0      & \dots  & 0      \\
		0      & a_{22} & \dots  & 0      \\
		\vdots & \vdots & \ddots & \vdots \\
		0      & 0      & \dots  & a_{nn}
	\end{pmatrix},
	\qquad
	\mathcal{B} = -\begin{pmatrix}
		0      & a_{12} & \dots  & a_{1n} \\
		a_{21} & 0      & \dots  & a_{2n} \\
		\vdots & \vdots & \ddots & \vdots \\
		a_{n1} & a_{n2} & \dots  & 0
	\end{pmatrix}.
$$

Оператор $A$ будем называть {\em невозмущенным} оператором, оператор $B$ {\em возмущением} оператора $A$, а $\mathbb{A}$ --- возмущенным линейным оператором. Как уже говорилось, элементы матрицы возмущения $\mathcal{B}$ малы по сравнению с элементами невозмущенной матрицы $\mathcal{A}$.

Требуется получить оценку одного определенного собственного значения (далее всюду будем делать вывод для первого собственного значения). Знаем, что это оцениваемое собственное значение отлично от других.

Разложим пространство $\spaceend\field{C}^n$ в прямую сумму $\mathfrak{X}_1 \oplus \mathfrak{X}_2$ инвариантных относительно невозмущенного оператора $A$ подпространств $\mathfrak{X}_1$ и $\mathfrak{X}_2$, где $\mathfrak{X}_1 = \field{C}, \mathfrak{X}_2 = \field{C}^{n-1}$. При этом потребуем, чтобы множества $\sigma_i = \sigma(A_i),\\ i=1,2$ взаимно не пересекались ($A_i = \left. A \right|_{\mathfrak{X}_i}, i=1,2$ --- сужение $A$ на $\mathfrak{X}_i$ и $A = A_1 \oplus A_2$). Как уже было сказано, будем искать оценку первого собственного значения, поэтому пусть $\mathfrak{X}_1 = \mathcal{L}(e_1)$ --- линейная оболочка, натянутая на базисный вектор $e_1 = (1,0, \dots, 0)$, а $\mathfrak{X}_2 = \mathcal{L}(e_2, \dots, e_n)$.

В соответствии с заданным разложением пространства $\field{C}^n$ будем рассматривать два трансформатора: $\mathcal{J}\colon \spaceend\field{C}^n \to \spaceend\field{C}^n, \Gamma\colon \spaceend\field{C}^n \to \spaceend\field{C}^n$, таких что:
\begin{enumerate}
	\item Для любого $X \in \spaceend\field{C}^n$ матрица оператора $\mathcal{J}X$ имеет вид:
	$$
		\mathcal{J}X = \begin{pmatrix}
			x_{11} & 0      & \dots  & 0      \\
			0      & x_{22} & \dots  & x_{2n} \\
			\vdots & \vdots & \ddots & \vdots \\
			0      & x_{n2} & \dots  & x_{nn}
		\end{pmatrix};
	$$
	\item $\Gamma X$ определяется как решение уравнения:
	\begin{equation}\label{eq:gamma_x_rule}
		A\Gamma X - \Gamma X A = X - \mathcal{J}X,\qquad \forall X \in \spaceend\field{C}^n.
	\end{equation}
\end{enumerate}

Определим вид матрицы оператора $\Gamma X$. Для этого запишем равенство~(\ref{eq:gamma_x_rule}) для элемента $(i,j)$:
\begin{equation}\label{eq:gamma_x_elements1}
	(A\Gamma X)_{ij} - (\Gamma X A)_{ij} = \begin{cases}
		x_{ij},\qquad i=1 \text{ и } j=2 \dots n; \\
		x_{ij},\qquad j=1 \text{ и } i=2 \dots n; \\
		0, \qquad \text{ иначе}.
	\end{cases}
\end{equation}
Пусть $\Gamma X = Y$. Заметим, что $(AY)_{ij} = a_{ii}y_{ij} $. Подставим полученный результат в~формулу~(\ref{eq:gamma_x_elements1}):
\begin{equation}\label{eq:gamma_x_elements2}
	a_{ii}y_{ij} - a_{jj}y_{ij} = \begin{cases}
		x_{ij},\qquad i=1 \text{ и } j=2 \dots n; \\
		x_{ij},\qquad j=1 \text{ и } i=2 \dots n; \\
		0, \qquad \text{ иначе}.
	\end{cases}
\end{equation}
Обозначим за $\Omega$ множество $ \{ (i,j)\colon i=1 \text{ и } j = 2 \dots n, \text{ или } j=1 \text{ и }\\ i = 2 \dots n \}$. Тогда
\begin{equation}\label{eq:gamma_x_elements2}
	y_{ij} = \begin{cases}
		\frac{x_{ij}}{a_{ii} - a_{jj}},\qquad (i,j) \in \Omega \\
		0, \qquad\qquad \text{ иначе}.
	\end{cases}
\end{equation}
Таким образом мы получили, что матрица оператора $\Gamma X$ имеет вид:
$$
	\Gamma X = \begin{pmatrix}
		0                              & \frac{x_{12}}{a_{11}-a_{22}}  & \dots  & \frac{x_{1n}}{a_{11}-a_{nn}}       \\
		\frac{x_{21}}{a_{22}-a_{11}}   & 0                             & \dots  & 0                                  \\
		\vdots                         & \vdots                        & \ddots & \vdots \\
		\frac{x_{n1}}{a_{nn}-a_{11}}   & 0                             & \dots  & 0
	\end{pmatrix}.
$$
Легко заметить, что $\norm{\Gamma X} \leqslant \gamma\norm{X}$, где $\gamma = \frac{1}{\min\limits_{\Omega} \abs{a_{ii} - a_{jj}}}$.
