
Пусть $P_1, P_2$ -- проекторы относительно разложения пространства $\mathfrak{X} = \mathfrak{X}_1 \oplus \mathfrak{X}_2$. 
Заметим, что $\forall X \in \spaceend\mathfrak{X}$ выполняется:
\begin{enumerate}
	\item $\mathcal{J}X = P_1 X P_1 + P_2 X P_2;$
	\item $P_i(\Gamma X)P_j = \Gamma(P_i X P_j),\ i,j = 1,2$, и $P_i(\Gamma X)P_i = 0,\ i = 1,2$.
\end{enumerate}
Таким образом пространство $\mathfrak{X}$ можно представить в виде прямой суммы $\mathfrak{X} = \mathfrak{X}_{11} \oplus \mathfrak{X}_{12} \oplus \mathfrak{X}_{21} \oplus \mathfrak{X}_{22}$ подпространств, где $\mathfrak{X}_{ij} = \{ P_i \mathfrak{X} P_j, X \in \spaceend\mathfrak{X} \},\\ \ i,j = 1,2$.
Через $X_{ij}$ будем обозначать оператор $P_i X P_j,\ i,j=1,2$, таким образом $X = (P_1 + P_2)X(P_1 + P_2) = X_{11} + X_{12} + X_{21} + X_{22},\ X \in \spaceend\mathfrak{X}$.
Применим операторы $P_1 \text{ и } P_2$ к обеим частям уравнения (\ref{eq:x_main}).
\begin{enumerate}
	\item Применим справа и слева проектор $P_1$:
	\begin{align}
		P_1 X P_1 &= P_1 B\Gamma X P_1 - P_1\Gamma X \mathcal{J}(B\Gamma X)P_1 - P_1 B P_1; \notag \\
		\begin{split}
			X_{11} &= (B_{11} + B_{12})(\Gamma X_{11} + \Gamma X_{21}) - \\
			&- (\Gamma X_{11} + \Gamma X_{12})\mathcal{J}((B_{11} + B_{12})(\Gamma X_{11} + \Gamma X_{21})) + B_{11}; \notag
		\end{split}
		\intertext{Будем учитывать, что $\mathcal{J}X_{12} = \mathcal{J}X_{21} = 0$, $\Gamma X_{11} = \Gamma X_{22} = 0$ 
		и $B_{11} = B_{22} = 0$.}
		X_{11} &= B_{12}\Gamma X_{21};\label{eq:x11}
	\end{align}

	\item Применим справа проектор $P_1$, а слева $P_2$:
	\begin{align}
		P_2 X P_1 &= P_2 B\Gamma X P_1 - P_2\Gamma X \mathcal{J}(B\Gamma X)P_1 - P_2 B P_1; \notag \\
		X_{21} &= (B_{11} + B_{22})\Gamma X_{21} - \Gamma X_{21}\mathcal{J}(B_{12}\Gamma X_{21}) - \Gamma X_{21}B_{11} + B_{21}; \notag \\
		X_{21} &= - (\Gamma X_{21})B_{12}\Gamma X_{21} + B_{21}; \label{eq:x21}
	\end{align}

\end{enumerate}

Искомую оценку элемента $x_{11}$ оператора $X$, являющегося решением нелинейного уравнения (\ref{eq:x_main}) мы получим, получив оценку $\norm{X_{11}}$. Для оценки $\norm{X_{11}}$ в свою очередь требуется оценка $\norm{X_{21}}$ и разрешимость уравнения (\ref{eq:x21}). Потому сформулируем и докажем следующую теорему:

\noindent\textbf{Теорема~2.}
{ \it Пусть выполнено неравенство
\begin{align}
d = 2\gamma^2 (b_{12}b_{21})^{\frac{1}{2}} < 1. \label{coef}
\end{align}
Тогда нелинейное уравнение \eqref{eq:x21} имеет единственное решение, которое можно найти методом простых итераций, и имеют место следующие оценки:
\begin{align*}
&\norm{X_{11} - B_{11}} \leq \frac{2\gamma b_{12}b_{21}}{1 + (1 - 4\gamma^2 b_{21}b_{12})^{\frac{1}{2}}}
&\norm{X_{21}} \leq \frac{2b_{21}}{1 + (1 - 4\gamma^2 b_{21}b_{12})^{\frac{1}{2}}},
\end{align*}
где $\norm{B_{ij}} = b_{ij},\ i,j = 1,2.$}

\noindent\textbf{Доказательство.}

Рассмотрим оператор $\Phi_1(X_{21}),$ определяемый уравнением \ref{eq:x21}. Найдем шар $B(r_1) = \{ X \in  \mathfrak{X}_{21}: \norm{X} < r_1 \}$ из пространства $ \mathfrak{X}_{21},$ который оператор $\Phi_1(X_{21})$ переводит в себя, т.\! е. $\norm{\Phi_1(X_{21})} \leq r_1$ для любого $X \in B(r_1).$ Обозначим $r_1 = rb_{21}.$
\begin{align*}
&\norm{\Phi_1(X_{21})} = - (\Gamma X_{21})B_{12}\Gamma X_{21} + B_{21} \leq \\
&\leq \gamma^2 b_{12}\norm{X_{21}}^2 + b_{21} \leq \\ 
&\leq \gamma^2 b_{21}^2 b_{12}r^2 + b_{21} \leq b_{21}r
\end{align*}
Получаем неравенство:
$$
\gamma^2 b_{21}^2 b_{12}r^2 - r + 1 \leq 0.
$$
Покажем что квадратное уравнение относительно $r$
$$
\gamma^2 b_{21}^2 b_{12}r^2 - r + 1 = 0
$$
имеет хотя бы один действительный положительный корень. Воспользуемся тем, что $\gamma\norm{B} < \frac{1}{3}$ и $b_{ij} \leq \norm{B}.$
\begin{align*}
&D = 1 - 4\gamma^2 b_{21}b_{12}; \\
&1 - 4\gamma^2 b_{21}b_{12} > 1 - 4\cdot\frac{1}{9} = 0
\end{align*}
Получаем, что 
$$
	r_{1,2} = \frac{1 \pm (1 - 4\gamma^2 b_{21}b_{12})^{\frac{1}{2}}}{2\gamma^2 b_{21}b_{12}}
$$	
Рассмотрим решение $r = \frac{1 - (1 - 4\gamma^2 b_{21}b_{12})^{\frac{1}{2}}}{2\gamma^2 b_{21}b_{12}},$ докажем, что оно больше нуля.
\begin{align*}
\frac{1 - (1 - 4\gamma^2 b_{21}b_{12})^{\frac{1}{2}}}{2\gamma^2 b_{21}b_{12}} = \frac{2}{1 + (1 - 4\gamma^2 b_{21}b_{12})^{\frac{1}{2}}} > 0
\end{align*}
Получаем, что в качестве радиуса шара можно взять число
$$
 rb_{21} = \frac{2b_{21}}{1 + (1 - 4\gamma^2 b_{21}b_{12})^{\frac{1}{2}}}.
$$
Для любой пары операторов $Y_1, Y_2$ из шара $B(r_1)$ имеет место оценка
\begin{align*}
&\norm{\Phi_1(Y_1) - \Phi_1(Y_2)} = \|- (\Gamma Y_1)B_{12}\Gamma Y_1 + B_{21} + \\
& + (\Gamma Y_2)B_{12}\Gamma Y_2 - B_{21}\| \leq (\gamma^2 b_{12}(\norm{Y_1} + \\
& + \norm{Y_2}))\norm{Y_1 - Y_2} \leq (\frac{2\gamma^2 (b_{12}b_{21})^{\frac{1}{2}}}{1 + (1 - 4\gamma^2 b_{21}b_{12})^{\frac{1}{2}}})\norm{Y_1 - Y_2} \leq d\norm{Y_1 - Y_2}.
\end{align*}
Из условия \eqref{coef} следует, что оператор $\Phi_1$ является оператором сжатия в шаре $B(r_1).$ Тогда уравнение \eqref{eq:x21} имеет единственное решение $X_{21}$ в этом шаре, которое можно найти методом простых итераций. Так как $X_{21}$ принадлежит шару $B(r_1),$ то справедливо неравенство
\begin{align*}
\norm{X_{11} - B_{11}} = \norm{B_{12}\Gamma X_{21}} \leq \gamma b_{12} \norm{X_{21}} \leq \frac{2\gamma b_{12}b_{21}}{1 + (1 - 4\gamma^2 b_{21}b_{12})^{\frac{1}{2}}}.
\end{align*}
Теорема доказана.

Таким образом мы имеем следующую оценку для $\norm{X_{11}}$
$$
	\norm{X_{11}} \leq \frac{2\gamma b_{12}b_{21}}{1 + (1 - 4\gamma^2 b_{21}b_{12})^{\frac{1}{2}}} \leq \frac{1}{2\gamma} - \frac{1 - (1 - 4\gamma^2 b_{21}b_{12})^{\frac{1}{2}}}{4\gamma^2 b_{21}b_{12}},
$$
где $\gamma = \norm{(\alpha I - B)^{-1}}, b_{ij} = \norm{B_{ij}}, i,j=1,2$ 

%Выпишем формулу оценки собственного значения $\lambda_1$ для разный норм вектора $x \in \field{C}^n.$

%Пусть норма вектора $x \in \field{C}^n$ равна
%$$
%	\norm{x} = \max\limits_{1 \leqslant i \leqslant n}\abs{x_i},
%$$ 
%которая в свою очередь порождает норму матриц соответствующих операторам $X \in \spaceend\field{C}^n$
%$$
%	\norm{X} = \max\limits_{1 \leqslant j \leqslant n}\sum\limits_{i = 1}^{n} \abs{x_{ij}}.
%$$
%Учитывая, что по условию нашей задачи $b_{11}$ равен 0, получаем следующую оценку $\lambda_1$
%$$
%	\abs{\lambda_1 - {a_{11}}} \leq \frac{ 2\gamma \max\limits_{2 \leqslant j \leqslant n}\abs{a_{1j}} \sum\limits_{i = 2}%^{n} \abs{a_{i1}} }{ 1 - \gamma \max\limits_{2 \leqslant j \leqslant n}\sum\limits_{i = 2}^{n} \abs{a_{ij}} + ((\gamma \max\limits_{2 \leqslant j \leqslant n}\sum\limits_{i = 2}^{n} \abs{a_{ij}} - 1)^2 - 4\gamma \max\limits_{2 \leqslant j \leqslant n}\abs{a_{1j}} \sum\limits_{i = 2}^{n} \abs{a_{i1}})^{\frac{1}{2}} }.
%$$

%Далее пусть норма вектора $x \in \field{C}^n$ равна
%$$
%	\norm{x} = \sum\limits_{i = 1}^{n}\abs{x_i},
%$$
%она порождает норму матриц соответствующих операторам $X \in \spaceend\field{C}^n$
%$$
%	\norm{X} = \max\limits_{1 \leqslant i \leqslant n}\sum\limits_{j = 1}^{n} \abs{x_{ij}}.
%$$ 
%Тогда получаем, следующую оценку
%$$
%	\abs{\lambda_1 - {a_{11}}} \leq \frac{ 2\gamma \sum\limits_{j = 2}^{n}\abs{a_{1j}} \max\limits_{2 \leqslant i \leqslant %n}\abs{a_{i1}} }{ 1 - \gamma \max\limits_{2 \leqslant i \leqslant n}\sum\limits_{j = 2}^{n} \abs{a_{ij}} + ((\gamma \max\limits_{2 \leqslant i \leqslant n}\sum\limits_{j = 2}^{n} \abs{a_{ij}} - 1)^2 - 4\gamma \sum\limits_{j = 2}^{n}\abs{a_{1j}} \max\limits_{2 \leqslant i \leqslant n}\abs{a_{i1}})^{\frac{1}{2}} }.
%$$