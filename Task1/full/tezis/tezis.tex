\documentclass{article}
\usepackage[russian]{babel}
\usepackage[utf8]{inputenc}
\usepackage{graphicx}
\usepackage{amsmath}
\usepackage{amsfonts}
\usepackage{amssymb}
\pagestyle{empty} \textwidth=108mm \textheight=165mm
\usepackage{calc,ifthen}

\newcommand{\me}[1]{#1\index{#1}}

\newcounter{first}

\newcommand{\tezis}[5][1]{\vbox{\protect\setcounter{first}{\thepage}%
\begin{center}\textbf{#2}%
\ifthenelse{\not\equal{#3}{}}{\footnotemark[#1]}{}%
\\*\textbf{#4}\\*%
\ifthenelse{\not\equal{#5}{}}{\textit{#5}}{}%
\end{center}\setcounter{equation}{0}%
}\ifthenelse{\not\equal{#3}{}}{\footnotetext[#1]{#3}}{}%
\nopagebreak[4]{}}
\multlinegap=0in


\begin{document}

\tezis{ЖОРДАНОВА ФОРМА ДЛЯ ЛОДУ N-ГО ПОРЯДКА}  % Заголовок прописными буквами
{}% Если нет, то поле оставить пустым {}
{\me{ Нестеров И.Н., Клочков С.В., Чурсанова А.С.} (Воронеж)} 
{nesterovilyan@gmail.com, klochkov\_s.v@mail.ru, anastasyachursanova@gmail.com}



%%%%%% Текст тезисов  %%%%%%%

Рассматривается линейное однородное дифференциальное уравнение~\ 
$$
	x^{(n)} = \alpha_{1}x^{(n-1)} + \ldots + \alpha_{n}, 
$$
где $\alpha_{k} \in \mathbb{C}, k = \overline{1,n}$. Данное уравнение обычным способом сводится к системе
линейных дифференциальных уравнений вида~\ $\dot{y} = Ay$ [1],
где матрица оператора $A$ имеет вид:
$$
	\mathcal{A} = \begin{pmatrix}
		0 & 1 & 0 & \dots & 0 \\
		0 & 0 & 1 & \dots & 0 \\
		\vdots & \vdots & \vdots & \ddots & \vdots \\
		0 & 0 & 0 & \dots & 1 \\
		\alpha_n & \alpha_{n-1} & \alpha_{n-1} & \dots & \alpha_1
	\end{pmatrix},
$$

\textbf{Теорема~1.}
{ \it Пусть $\lambda_1, \ldots , \lambda_m$ собственные значения матрицы $\mathcal{A}$ кратностей $k_1, \ldots , k_m$ соответственно. Тогда жорданова форма матрицы $\mathcal{A}$ имеет вид:
$$
	\mathcal{J} = \begin{pmatrix}
		\mathcal{J}_1 & 0 & \dots & 0 \\
		0 & \mathcal{J}_2 & \dots & 0 \\
		\vdots & \vdots & \ddots & \vdots \\
		0 & 0 & \dots & \mathcal{J}_m
	\end{pmatrix}, {\text{где }}
	\mathcal{J}_i = \begin{pmatrix}
		\lambda_i & 1 & 0 & \dots & 0 \\
		0 & \lambda_i & 1 & \dots & 0 \\
		\vdots & \vdots & \vdots & \ddots & \vdots \\
		0 & 0 & 0 & \dots & 1 \\
		0 & 0 & 0 & \dots & \lambda_i
	\end{pmatrix}.
$$
Векторами столбцами матрицы перехода являются собственные векторы 
$e_i = (1, \lambda_i, \ldots, \lambda_i^{n-1}) $ и присоединенные к ним вида: \\
$
	(0, 1 \ldots, (n-1)\lambda_i^{n-2}),\ldots,(\underbrace{0, \ldots,0}_{k_i-1},
	(k_i-1)!,\ldots, \frac{(n-1)!}{(n-k_i)!}\lambda_i^{n-k_i}), i=\overline{1,m}
$

\end{document}
