\documentclass[10pt,a4paper,oneside]{article}
\usepackage[T2A]{fontenc}
\usepackage[utf8]{inputenc}
\usepackage[russian]{babel}
\usepackage{mathtools}
\usepackage{amssymb}
\usepackage{setspace}
\usepackage{amsthm}


\begin{document}
 
\begin{center}
    \setstretch{1.5}
    \textbf{\LARGE ЖОРДАНОВА ФОРМА ДЛЯ СОПРОВОЖДАЮЩИХ МАТРИЦ} \\[1em]
    \textbf{И.\,Н.\,Нестеров,\,С.\,В.\,Клочков,\,А.\,С.\,Чурсанова} \\[2em]
\end{center}

Пусть $\mathfrak{X}_1$, $\mathfrak{X}_2$ -- два комплексных банаховых пространства. Символом 
$\operatorname{Hom}(\mathfrak{X}_1, \mathfrak{X}_2)$ обозначим банахово пространство линейных ограниченных \\ операторов, определенных на $\mathfrak{X}_1$ со значениями в $\mathfrak{X}_2$. Через 
$\operatorname{End}\mathfrak{X}$ обозначим банахово пространство $\operatorname{Hom}(\mathfrak{X}, \mathfrak{X})$, 
если $\mathfrak{X}_1 = \mathfrak{X}_2 = \mathfrak{X}$.

Рассмотрим линейный ограниченный оператор $\mathbb{A} \in \spaceend\mathfrak{X},$
где $\mathfrak{X}$ банахово пространство, являющееся прямой суммой 
$\mathfrak{X} = \mathfrak{X}_1 \oplus \mathfrak{X}_2$ двух замкнутых подпространств 
$\mathfrak{X}_1, \mathfrak{X}_2$. Предполагается, что оператор $\mathbb{A}$ задается операторной матрицей
$$
\mathbb{A} = \begin{pmatrix}
		A & C \\
		D & B
	\end{pmatrix},
$$
т.е. $\mathbb{A}x = (A x_1 + C x_2, D x_1 + B x_2)$, где $x = (x_1, x_2) \in \mathfrak{X}_1 \times \mathfrak{X}_2$. Отметим, что

\begin{align*}
A \in \operatorname{End}\mathfrak{X}_1, \ C \in \operatorname{Hom}(\mathfrak{X}_2, \mathfrak{X}_1), \\
B \in \operatorname{End}\mathfrak{X}_2, \ D \in \operatorname{Hom}(\mathfrak{X}_1, \mathfrak{X}_2).
\end{align*}
Представим оператор $\mathbb{A}$ в виде $\mathbb{A} = \mathcal{A} - \mathcal{B}$, где оператор 
$\mathcal{A} \in \spaceend(\mathfrak{X}_1 \times \mathfrak{X}_2)$ задается матрицей 
$\begin{pmatrix}
		A & 0 \\
		0 & B
\end{pmatrix}, $ а оператор $\mathcal{B} \in \spaceend(\mathfrak{X}_1 \times \mathfrak{X}_2)$ -- матрицей
$\begin{pmatrix}
		0 & -C \\
		-D & 0
\end{pmatrix}.$
Всюду далее считаем что выполняется условие:
$$
\sigma(A) \cap \sigma(B) = {\varnothing}.
$$

Символом $\mathcal{U}$ обозначим пространство $\spaceend(\mathfrak{X}_1 \times \mathfrak{X}_2)$ Рассмотрим канонические проекторы
$$
P_1x = (x_1, 0), P_2x = (0, x_2), x = (x_1, x_2) \in \mathfrak{X}_1 \times \mathfrak{X}_2.
$$
Для любого оператора $X \in \mathcal{U}$ рассмотрим операторы $P_iXP_j \in \mathcal{U}, i,j \in {1,2}.$ Таким образом, оператор $X$ задается матрицей:
$$
X = \begin{pmatrix}
		X_{11} & X_{12} \\
		X_{21} & X_{22}
	\end{pmatrix},
$$
где оператор $X_{ij}$ -- сужение оператора $P_iXP_j$ на подпространство $\mathfrak{X}_i$ с областью значений $\mathfrak{X}_j, i,j \in {1,2}.$

В соответствии с заданным разложением пространства $\mathfrak{X}$ будем рассматривать два трансформатора: $\mathcal{J} \in \spaceend\mathcal{U}, \Gamma \in \spaceend\mathcal{U}$, таких что:
\begin{enumerate}
	\item Для любого $X \in \mathcal{U}$ оператор $\mathcal{J}X$ определяется следующим образом: $\mathcal{J}X = P_1XP_1 + P_2XP_2,$ а матрица оператора $\mathcal{J}X$ имеет вид:
	$$
	\mathcal{J}X = \begin{pmatrix}
		X_{11} & 0 \\
		0 & X_{22}
	\end{pmatrix};
	$$
	\item Трансформатор $\Gamma \in \spaceend\mathcal{U}$ построим следующим образом. Оператор 
	$\Gamma X$, где $X \in \mathcal{U}$ определим как решение $Y \in \spaceend\mathcal{U}$ уравнения
	\begin{equation}\label{eq:gamma_x_rule}
		\mathcal{A}Y - Y\mathcal{A} = X - \mathcal{J}X,\qquad \forall X \in \mathcal{U},
	\end{equation}
	удовлетворяющее условию $\mathcal{J}\mathcal{Y} = 0$, кроме того можно показать, что решение 
	$Y$ существует и единственно.
\end{enumerate}
Запишем уравнение \eqref{eq:gamma_x_rule} в матричном виде:
$$
\begin{pmatrix}
		A & 0 \\
		0 & B
\end{pmatrix}
\begin{pmatrix}
		0 & Y_{12} \\
		Y_{21} & 0
\end{pmatrix} -
\begin{pmatrix}
		0 & Y_{12} \\
		Y_{21} & 0
\end{pmatrix}
\begin{pmatrix}
		A & 0 \\
		0 & B
\end{pmatrix} =
\begin{pmatrix}
		0 & X_{12} \\
		X_{21} & 0
\end{pmatrix}.	 	
$$
Перемножив и вычтя матрицы получим следующую систему операторныx уравнений:
\begin{equation}\label{syst:gamma_x_rule}
	\begin{cases}
		AY_{12} - Y_{12}B = X_{12}, \\
		BY_{21} - Y_{21}A = X_{21},
	\end{cases}
\end{equation}
где $Y_{12}, Y_{21}$ -- искомые операторы. Если оператор $A$ или оператор $B$ ограничен, то уравнения \eqref{syst:gamma_x_rule} разрешимы.

Рассмотрим случай, когда $\dim\mathfrak{X}_1 = 1,$ т.\! е. оператор $A$ -- скалярный оператор: $A = \alpha I.$ Перепишем уравнения \ref{syst:gamma_x_rule}:
$$
	\begin{cases}
		\alpha Y_{12} - Y_{12}B = X_{12}, \\
		BY_{21} - \alpha Y_{21} = X_{21}.
	\end{cases}
$$
Так как оператор $A$ ограничен, то система имеет решение, и оно имеет вид:
$$
	\begin{cases}
		Y_{12} = X_{12}(\alpha I - B)^{-1}, \\
		Y_{21} = (\alpha I - B)^{-1}X_{21}.
	\end{cases}
$$
Таким образом мы получили, что матрица оператора $\Gamma X$ имеет вид:
$$
	\Gamma X = \begin{pmatrix}
		0 & X_{12}(\alpha I - B)^{-1} \\
		(\alpha I - B)^{-1}X_{21} & 0
	\end{pmatrix}.
$$
Оценим норму оператора $\Gamma.$ Пусть $X \in \mathcal{U},$ тогда:
\begin{align*}
&\norm{\Gamma X} \leqslant \max{(\norm{X_{12}(\alpha I - B)^{-1}}, \norm{(\alpha I - B)^{-1}X_{21}})} \leqslant \\ 
&\leqslant \max{(\norm{(\alpha I - B)^{-1}}, \norm{(\alpha I - B)^{-1}})}\norm{X} = \norm{(\alpha I - B)^{-1}}\norm{X} = \gamma \norm{X},
\end{align*}  
где $\gamma = \norm{(\alpha I - B)^{-1}}$.
Итак, получена следующая оценка $\norm{\Gamma} \leqslant \gamma$.

\section{Теорема}
\textbf{Теорема~1.}
{ \it Пусть $\lambda_1, \ldots , \lambda_m$ собственные значения матрицы $\mathcal{A}$ кратностей $k_1, \ldots , k_m$ соответственно. Тогда жорданова форма для матрицы $\mathcal{A}$ имеет вид:
$$
	\mathcal{J} = \begin{pmatrix}
		\mathcal{J}_1 & 0 & \dots & 0 \\
		0 & \mathcal{J}_2 & \dots & 0 \\
		\vdots & \vdots & \ddots & \vdots \\
		0 & 0 & \dots & \mathcal{J}_m
	\end{pmatrix}, {\text{где }}
	\mathcal{J}_i = \begin{pmatrix}
		\lambda_i & 1 & 0 & \dots & 0 \\
		0 & \lambda_i & 1 & \dots & 0 \\
		\vdots & \vdots & \vdots & \ddots & \vdots \\
		0 & 0 & 0 & \dots & 1 \\
		0 & 0 & 0 & \dots & \lambda_i
	\end{pmatrix}.
$$
Векторами столбцами матрицы перехода являются собственные и присоединенные векторы, где собственные векторы имеют вид: 
$$x^i =(1, \lambda_i, \ldots, \lambda_i^{n-1}), 1 \leqslant i \leqslant m,$$
а присоединенные:
$$y_i^j = (1^{(j)}, \lambda_i^{(j)}, \ldots , (\lambda_i^{n-1})^{(j)}), 1 \leqslant i \leqslant m, 1 \leqslant j \leqslant k_i.$$} 




В качестве примера рассмотрим частный случай~$n=3$. Тогда матрица~$\mathcal{A}$ имеет вид:
$$
 \mathcal{A} = 
 \begin{pmatrix}
  0 & 1 & 0\\
  0 & 0 & 1 \\
  \alpha_3 & \alpha_2 & \alpha_1 \\
 \end{pmatrix}
$$
Возможны три варианта: \\
%\begin{enumerate}
1) все собственные значения различны; \\
2) собственные значения $\lambda_1, \lambda_2$ имеют кратности ${k_1=2},~{k_2=1}$ (либо наоборот); \\
3) матрица $\mathcal{A}$ имеет одно собственное значение $\lambda_0$;        
%\end{enumerate}

\textbf{Рассмотрим первый случай.} 
Пусть матрица $\mathcal{A}$ имеет собственные значения $\lambda_1, \lambda_2, \lambda_3$ кратностей~ $k_1=k_2=k_3=1$ соответственно. Оператор $A$ имеет простую структуру. Тогда жорданова форма матрицы~$\mathcal{A}$ имеет вид:
$$
 \mathcal{J} = 
 \begin{pmatrix}
  \lambda_1 & 0 & 0\\
  0 & \lambda_2 & 0 \\
  0 & 0 & \lambda_3 \\
 \end{pmatrix}
$$
Далее запишем матрицу перехода~$\mathcal{U}$ и~обратную к~ней~$\mathcal{U}^{-1}$:
$$
\begin{aligned}
 \mathcal{U} &= 
 \begin{pmatrix}
  	1 & 1 & 1\\
  	\lambda_1 & \lambda_2 & \lambda_3 \\
  	\lambda_1^2 & \lambda_2^2 & \lambda_3^2 \\
 \end{pmatrix}, \\
 \mathcal{U}^{-1} &= \frac{1}{\Delta} 
 \begin{pmatrix}
  	\lambda_2\lambda_3(\lambda_3-\lambda_2) & \lambda_2^2-\lambda_3^2 & \lambda_3-\lambda_2\\
  	\lambda_1\lambda_3(\lambda_1-\lambda_3) & \lambda_3^2-\lambda_1^2 & \lambda_1-\lambda_3\\
  	\lambda_1\lambda_2(\lambda_2-\lambda_1) & \lambda_1^2-\lambda_2^2 & \lambda_2-\lambda_1\\
 \end{pmatrix},
\end{aligned}
$$
где $\Delta = \det \mathcal{U} = (\lambda_3-\lambda_1)(\lambda_3-\lambda_2)(\lambda_2-\lambda_1).$ \\
Рассмотрим проекторы $\mathcal{P}_i'$ жордановой матрицы~$\mathcal{J}$ оператора~$A$, которые имеют вид:
$$
\begin{aligned}
\mathcal{P}_i' &= \left(p_{jk}'\right), \\
		p_{jk}' &= 
		\begin{cases}
			1, & \text{если $j=k=i$;} \\
			0, & \text{в противном случае.}
		\end{cases}
\end{aligned}
$$
где $i,j,k \in \{1,2,3\}$.\\
Теперь запишем проекторы оператора~$A$:
$$
	\begin{aligned}
		\mathcal{P}_i &= \mathcal{U}\mathcal{P}_i'\mathcal{U}^{-1}, 
	\end{aligned}
$$
где $i \in \{1,2,3\}$. \\
Найдем проектор $\mathcal{P}_1$
$$
\mathcal{P}_1 = 
\frac{1}{\Delta}
\begin{pmatrix}
  	1 & 1 & 1\\
  	\lambda_1 & \lambda_2 & \lambda_3 \\
  	\lambda_1^2 & \lambda_2^2 & \lambda_3^2 \\
 \end{pmatrix}
\begin{pmatrix}
  	1 & 0 & 0\\
  	0 & 0 & 0\\
  	0 & 0 & 0 \\
 \end{pmatrix}
\begin{pmatrix}
  	\lambda_2\lambda_3(\lambda_3-\lambda_2) & \lambda_2^2-\lambda_3^2 & \lambda_3-\lambda_2\\
  	\lambda_1\lambda_3(\lambda_1-\lambda_3) & \lambda_3^2-\lambda_1^2 & \lambda_1-\lambda_3\\
  	\lambda_1\lambda_2(\lambda_2-\lambda_1) & \lambda_1^2-\lambda_2^2 & \lambda_2-\lambda_1\\
 \end{pmatrix},
$$
$$
\mathcal{P}_1 = 
\frac{1}{\Delta}
\begin{pmatrix}
  	1 & 0 & 0\\
  	\lambda_1 & 0 & 0 \\
  	\lambda_1^2 & 0 & 0 \\
 \end{pmatrix}
\begin{pmatrix}
  	\lambda_2\lambda_3(\lambda_3-\lambda_2) & \lambda_2^2-\lambda_3^2 & \lambda_3-\lambda_2\\
  	\lambda_1\lambda_3(\lambda_1-\lambda_3) & \lambda_3^2-\lambda_1^2 & \lambda_1-\lambda_3\\
  	\lambda_1\lambda_2(\lambda_2-\lambda_1) & \lambda_1^2-\lambda_2^2 & \lambda_2-\lambda_1\\
 \end{pmatrix},
$$
$$
\mathcal{P}_1 = 
\frac{1}{\Delta}
\begin{pmatrix}
  	\lambda_2\lambda_3(\lambda_3-\lambda_2) & \lambda_2^2-\lambda_3^2 & \lambda_3-\lambda_2\\
  	\lambda_1\lambda_2\lambda_3(\lambda_3-\lambda_2) & \lambda_1\left(\lambda_2^2-\lambda_3^2\right) & \lambda_1\left(\lambda_3-\lambda_2\right)\\
  	\lambda_1^2\lambda_2\lambda_3(\lambda_3-\lambda_2) & \lambda_1^2\left(\lambda_2^2-\lambda_3^2\right) & \lambda_1^2\left(\lambda_3-\lambda_2\right)\\
 \end{pmatrix},
$$
где $\Delta = (\lambda_3-\lambda_1)(\lambda_3-\lambda_2)(\lambda_2-\lambda_1).$ \\
Найдем проектор $\mathcal{P}_2$
$$
\mathcal{P}_2 = 
\frac{1}{\Delta}
\begin{pmatrix}
  	1 & 1 & 1\\
  	\lambda_1 & \lambda_2 & \lambda_3 \\
  	\lambda_1^2 & \lambda_2^2 & \lambda_3^2 \\
 \end{pmatrix}
\begin{pmatrix}
  	0 & 0 & 0\\
  	0 & 1 & 0\\
  	0 & 0 & 0 \\
 \end{pmatrix}
\begin{pmatrix}
  	\lambda_2\lambda_3(\lambda_3-\lambda_2) & \lambda_2^2-\lambda_3^2 & \lambda_3-\lambda_2\\
  	\lambda_1\lambda_3(\lambda_1-\lambda_3) & \lambda_3^2-\lambda_1^2 & \lambda_1-\lambda_3\\
  	\lambda_1\lambda_2(\lambda_2-\lambda_1) & \lambda_1^2-\lambda_2^2 & \lambda_2-\lambda_1\\
 \end{pmatrix},
$$
$$
\mathcal{P}_2 = 
\frac{1}{\Delta}
\begin{pmatrix}
  	0 & 1 & 0\\
  	0 & \lambda_2 & 0 \\
  	0 & \lambda_2^2 & 0 \\
 \end{pmatrix}
\begin{pmatrix}
  	\lambda_2\lambda_3(\lambda_3-\lambda_2) & \lambda_2^2-\lambda_3^2 & \lambda_3-\lambda_2\\
  	\lambda_1\lambda_3(\lambda_1-\lambda_3) & \lambda_3^2-\lambda_1^2 & \lambda_1-\lambda_3\\
  	\lambda_1\lambda_2(\lambda_2-\lambda_1) & \lambda_1^2-\lambda_2^2 & \lambda_2-\lambda_1\\
 \end{pmatrix},
$$
$$
\mathcal{P}_2 = 
\frac{1}{\Delta}
\begin{pmatrix}
  	\lambda_1\lambda_3(\lambda_1-\lambda_3) & \lambda_3^2-\lambda_1^2 & \lambda_1-\lambda_3\\
  	\lambda_1\lambda_2\lambda_3(\lambda_1-\lambda_3) & \lambda_2(\lambda_3^2-\lambda_1^2) & \lambda_2(\lambda_1-\lambda_3)\\
  	\lambda_1\lambda_2^2\lambda_3(\lambda_1-\lambda_3) & \lambda_2^2(\lambda_3^2-\lambda_1^2) & \lambda_2^2(\lambda_1-\lambda_3)\\
 \end{pmatrix},
$$
где $\Delta = (\lambda_3-\lambda_1)(\lambda_3-\lambda_2)(\lambda_2-\lambda_1).$ \\
Найдем проектор $\mathcal{P}_3$
$$
\mathcal{P}_3 = 
\frac{1}{\Delta}
\begin{pmatrix}
  	1 & 1 & 1\\
  	\lambda_1 & \lambda_2 & \lambda_3 \\
  	\lambda_1^2 & \lambda_2^2 & \lambda_3^2 \\
 \end{pmatrix}
\begin{pmatrix}
  	0 & 0 & 0\\
  	0 & 0 & 0\\
  	0 & 0 & 1 \\
 \end{pmatrix}
\begin{pmatrix}
  	\lambda_2\lambda_3(\lambda_3-\lambda_2) & \lambda_2^2-\lambda_3^2 & \lambda_3-\lambda_2\\
  	\lambda_1\lambda_3(\lambda_1-\lambda_3) & \lambda_3^2-\lambda_1^2 & \lambda_1-\lambda_3\\
  	\lambda_1\lambda_2(\lambda_2-\lambda_1) & \lambda_1^2-\lambda_2^2 & \lambda_2-\lambda_1\\
 \end{pmatrix},
$$
$$
\mathcal{P}_3 = 
\frac{1}{\Delta}
\begin{pmatrix}
  	0 & 0 & 1\\
  	0 & 0 & \lambda_3 \\
  	0 & 0 & \lambda_3^2 \\
 \end{pmatrix}
\begin{pmatrix}
  	\lambda_2\lambda_3(\lambda_3-\lambda_2) & \lambda_2^2-\lambda_3^2 & \lambda_3-\lambda_2\\
  	\lambda_1\lambda_3(\lambda_1-\lambda_3) & \lambda_3^2-\lambda_1^2 & \lambda_1-\lambda_3\\
  	\lambda_1\lambda_2(\lambda_2-\lambda_1) & \lambda_1^2-\lambda_2^2 & \lambda_2-\lambda_1\\
 \end{pmatrix},
$$
$$
\mathcal{P}_3 = 
\frac{1}{\Delta}
\begin{pmatrix}
  	\lambda_1\lambda_2(\lambda_2-\lambda_1) & \lambda_1^2-\lambda_2^2 & \lambda_2-\lambda_1\\
  	\lambda_1\lambda_2\lambda_3(\lambda_2-\lambda_1) & \lambda_3(\lambda_1^2-\lambda_2^2) & \lambda_3(\lambda_2-\lambda_1)\\
  	\lambda_1\lambda_2\lambda_3^2(\lambda_2-\lambda_1) & \lambda_3^2(\lambda_1^2-\lambda_2^2) & \lambda_3^2(\lambda_2-\lambda_1)\\
 \end{pmatrix},
$$
где $\Delta = (\lambda_3-\lambda_1)(\lambda_3-\lambda_2)(\lambda_2-\lambda_1).$

Тогда по теореме о спектральном разложении оператора простой структуры [2]:
$$
	\mathcal{A} = \lambda_1\mathcal{P}_1 + \lambda_2\mathcal{P}_2 + \lambda_3\mathcal{P}_3 
$$

\textbf{Рассмотрим второй случай.} 		
Пусть матрица~$\mathcal{A}$ имеет собственные значения~$\lambda_1,~\lambda_2$ 
которые имеют кратности~$k_1=2,~k_2=1$ соответственно. Тогда жорданова форма матрицы~$\mathcal{A}$ имеет вид:
$$
 \mathcal{J} = 
 \begin{pmatrix}
  \lambda_1 & 1 & 0\\
  0 & \lambda_1 & 0 \\
  0 & 0 & \lambda_2 \\
 \end{pmatrix}
$$
Запишем матрицу перехода~$\mathcal{U}$ и~обратную к~ней~$\mathcal{U}^{-1}$:
$$
\begin{aligned}
 \mathcal{U} &= 
 \begin{pmatrix}
  	1 & 0 & 1\\
  	\lambda_1 & 1 & \lambda_2 \\
  	\lambda_1^2 & 2\lambda_1 & \lambda_2^2 \\
 \end{pmatrix}, \\
 \mathcal{U}^{-1} &= \frac{1}{\Delta} 
 \begin{pmatrix}
  	\lambda_2^2-2\lambda_1\lambda_2 & 2\lambda_1 & -1  \\
  	\lambda_1^2\lambda_2 - \lambda_1\lambda_2^2 & \lambda_2^2 - \lambda_1^2 & \lambda_1 - \lambda_2  \\
  	\lambda_1^2 & -2\lambda_1 & 1  \\
 \end{pmatrix},
\end{aligned}
$$
где $\Delta = \det \mathcal{U} = (\lambda_1-\lambda_2)^2$. \\
Рассмотрим проекторы $\mathcal{P}_i', i \in \{1,2\}$ жордановой матрицы~$\mathcal{J}$ оператора~$A$ и её нильпотентную часть $\mathcal{Q}'$.
$$
\begin{aligned}
\mathcal{P}_1' = 
		\begin{pmatrix}
  			1 & 0 & 0\\
  			0 & 1 & 0 \\
  			0 & 0 & 0 \\
 		\end{pmatrix}&,\  
 		\mathcal{P}_2' =
		\begin{pmatrix}
  			0 & 0 & 0\\
  			0 & 0 & 0 \\
  			0 & 0 & 1 \\
 		\end{pmatrix},\\
 		 \mathcal{Q}' &= 
		\begin{pmatrix}
  			0 & 1 & 0\\
  			0 & 0 & 0 \\
  			0 & 0 & 0 \\
 		\end{pmatrix}
\end{aligned}
$$
Теперь запишем проекторы оператора $A$ и нильпотентную часть жордановой матрицы:
$$
	\begin{aligned}
		\mathcal{P}_i &= \mathcal{U}\mathcal{P}_i'\mathcal{U}^{-1}, \\ 
 		\mathcal{Q} &= \mathcal{U}\mathcal{Q}'\mathcal{U}^{-1}, 
	\end{aligned}
$$
где $i \in \{1,2\}$. \\
Найдем проектор $\mathcal{P}_1$
$$
\mathcal{P}_1 = 
\begin{pmatrix}
  	1 & 0 & 1\\
  	\lambda_1 & 1 & \lambda_2 \\
  	\lambda_1^2 & 2\lambda_1 & \lambda_2^2 \\
\end{pmatrix}
 \begin{pmatrix}
  	1 & 0 & 0\\
  	0 & 1 & 0 \\
  	0 & 0 & 0 \\
\end{pmatrix}
\begin{pmatrix}
  	\lambda_2^2-2\lambda_1\lambda_2 & 2\lambda_1 & -1  \\
  	\lambda_1^2\lambda_2 - \lambda_1\lambda_2^2 & \lambda_2^2 - \lambda_1^2 & \lambda_1 - \lambda_2  \\
  	\lambda_1^2 & -2\lambda_1 & 1  \\
 \end{pmatrix} 		
$$
$$
\mathcal{P}_1 = 
\begin{pmatrix}
  	1 & 0 & 0\\
  	\lambda_1 & 1 & 0 \\
  	\lambda_1^2 & 2\lambda_1 & 0 \\
\end{pmatrix}
\begin{pmatrix}
  	\lambda_2^2-2\lambda_1\lambda_2 & 2\lambda_1 & -1  \\
  	\lambda_1^2\lambda_2 - \lambda_1\lambda_2^2 & \lambda_2^2 - \lambda_1^2 & \lambda_1 - \lambda_2  \\
  	\lambda_1^2 & -2\lambda_1 & 1  \\
 \end{pmatrix} 		
$$
$$
\mathcal{P}_1 = 
\begin{pmatrix}
  	\lambda_2^2-2\lambda_1\lambda_2 & 2\lambda_1 & -1  \\
  	-\lambda_1^2\lambda_2 & \lambda_1^2+\lambda_2^2  & - \lambda_2  \\
  	-\lambda_1^2\lambda_2^2 & 2\lambda_1\lambda_2^2 & \lambda_1^2-1\lambda_1\lambda_2  \\
 \end{pmatrix} 	
$$
Найдем проектор $\mathcal{P}_2$
$$
\mathcal{P}_2 = 
\begin{pmatrix}
  	1 & 0 & 1\\
  	\lambda_1 & 1 & \lambda_2 \\
  	\lambda_1^2 & 2\lambda_1 & \lambda_2^2 \\
\end{pmatrix}
 \begin{pmatrix}
  	0 & 0 & 0\\
  	0 & 0 & 0 \\
  	0 & 0 & 1 \\
\end{pmatrix}
\begin{pmatrix}
  	\lambda_2^2-2\lambda_1\lambda_2 & 2\lambda_1 & -1  \\
  	\lambda_1^2\lambda_2 - \lambda_1\lambda_2^2 & \lambda_2^2 - \lambda_1^2 & \lambda_1 - \lambda_2  \\
  	\lambda_1^2 & -2\lambda_1 & 1  \\
 \end{pmatrix} 	
$$
$$
\mathcal{P}_2 = 
\begin{pmatrix}
  	0 & 0 & 1\\
  	0 & 0 & \lambda_2 \\
  	0 & 0 & \lambda_2^2 \\
\end{pmatrix}
\begin{pmatrix}
  	\lambda_2^2-2\lambda_1\lambda_2 & 2\lambda_1 & -1  \\
  	\lambda_1^2\lambda_2 - \lambda_1\lambda_2^2 & \lambda_2^2 - \lambda_1^2 & \lambda_1 - \lambda_2  \\
  	\lambda_1^2 & -2\lambda_1 & 1  \\
 \end{pmatrix} 
$$
$$
\mathcal{P}_2 = 
\begin{pmatrix}
  	\lambda_1^2 & -2\lambda_1 & 1  \\
  	\lambda_1^2\lambda_2 & -2\lambda_1\lambda_2 & \lambda_2  \\
  	\lambda_1^2\lambda_2^2 & -2\lambda_1\lambda_2^2 & \lambda_2^2  \\
 \end{pmatrix} 
$$
Найдем нильпотентную часть жордановой матрицы $\mathcal{Q}$
$$
\mathcal{Q}=
\begin{pmatrix}
  	1 & 0 & 1\\
  	\lambda_1 & 1 & \lambda_2 \\
  	\lambda_1^2 & 2\lambda_1 & \lambda_2^2 \\
\end{pmatrix}
 \begin{pmatrix}
  	0 & 1 & 0\\
  	0 & 0 & 0 \\
  	0 & 0 & 0 \\
\end{pmatrix}
\begin{pmatrix}
  	\lambda_2^2-2\lambda_1\lambda_2 & 2\lambda_1 & -1  \\
  	\lambda_1^2\lambda_2 - \lambda_1\lambda_2^2 & \lambda_2^2 - \lambda_1^2 & \lambda_1 - \lambda_2  \\
  	\lambda_1^2 & -2\lambda_1 & 1  \\
 \end{pmatrix}
$$
$$
\mathcal{Q}=
 \begin{pmatrix}
  	0 & 1 & 0\\
  	0 & \lambda_1 & 0 \\
  	0 & \lambda_1^2 & 0 \\
\end{pmatrix}
\begin{pmatrix}
  	\lambda_2^2-2\lambda_1\lambda_2 & 2\lambda_1 & -1  \\
  	\lambda_1^2\lambda_2 - \lambda_1\lambda_2^2 & \lambda_2^2 - \lambda_1^2 & \lambda_1 - \lambda_2  \\
  	\lambda_1^2 & -2\lambda_1 & 1  \\
 \end{pmatrix}
$$
$$
\mathcal{Q}=
\begin{pmatrix}
  	\lambda_1^2\lambda_2 - \lambda_1\lambda_2^2 & \lambda_2^2 - \lambda_1^2 & \lambda_1 - \lambda_2  \\
  	\lambda_1(\lambda_1^2\lambda_2 - \lambda_1\lambda_2^2) & \lambda_1(\lambda_2^2 - \lambda_1^2) & \lambda_1(\lambda_1 - \lambda_2)  \\
  	\lambda_1^2(\lambda_1^2\lambda_2 - \lambda_1\lambda_2^2) & \lambda_1^2(\lambda_2^2 - \lambda_1^2) & \lambda_1^2(\lambda_1 - \lambda_2)  \\
 \end{pmatrix}
$$

Тогда по теореме о спектральном разложении линейного оператора [2]:
$$
	\mathcal{A} = \lambda_1\mathcal{P}_1 + \lambda_2\mathcal{P}_2 + \mathcal{Q}.
$$

\textbf{Рассмотрим третий случай.}	
Пусть матрица~$\mathcal{A}$ имеет одно собственное значение~$\lambda$ 
кратности~$k=3$. Тогда жорданова форма матрицы~$\mathcal{A}$ имеет вид:
$$
 \mathcal{J} = 
 \begin{pmatrix}
  \lambda & 1 & 0\\
  0 & \lambda & 1 \\
  0 & 0 & \lambda \\
 \end{pmatrix}
$$
Запишем матрицу перехода~$\mathcal{U}$ и~обратную к~ней~$\mathcal{U}^{-1}$:
$$
\begin{aligned}
 \mathcal{U} &= 
 \begin{pmatrix}
  	1 & 0 & 0\\
  	\lambda & 1 & 0 \\
  	\lambda^2 & 2\lambda & 2 \\
 \end{pmatrix}, \\
 \mathcal{U}^{-1} &= \frac{1}{2} 
 \begin{pmatrix}
  	1 & 0 & 0  \\
  	-\lambda & 1 & 0  \\
  	\frac{1}{2}\lambda^2 & -\lambda & \frac{1}{2} \\
 \end{pmatrix}.
\end{aligned}
$$
Спектральное разложение для этого случая очевидно.

\vfill

%%%%  ОФОРМЛЕНИЕ СПИСКА ЛИТЕРАТУРЫ %%%
\smallskip \centerline{\bf Литература}\nopagebreak

1. \textit{ Боровских А.В., Перов А.И.} Лекции по обыкновенным дифференциальным уравнениям // Москва-Ижевск: НИЦ  <<Регулярная и хаотическая динамика>>, Институт компьютерных исследований, 2004, 540 стр

2. \textit{ Баскаков А. Г.} Лекции по алгебре // Воронеж: Издательско-полиграфический центр Воронежского государственного университета, 2013, 159 стр

\end{document}
