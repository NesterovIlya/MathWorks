\textbf{Теорема~1.}
{ \it Пусть $\lambda_1, \ldots , \lambda_m$ -- собственные значения матрицы $\mathcal{A}$ кратностей $k_1, \ldots , k_m$ соответственно, где $\sum\limits_{i=1}^m k_i = n$. Тогда жорданова форма для матрицы $\mathcal{A}$ имеет вид
$$
	\mathcal{J} = \begin{pmatrix}
		\mathcal{J}_1 & 0 & \dots & 0 \\
		0 & \mathcal{J}_2 & \dots & 0 \\
		\vdots & \vdots & \ddots & \vdots \\
		0 & 0 & \dots & \mathcal{J}_m
	\end{pmatrix}, {\text{где }}
	\mathcal{J}_i = \begin{pmatrix}
		\lambda_i & 1 & 0 & \dots & 0 \\
		0 & \lambda_i & 1 & \dots & 0 \\
		\vdots & \vdots & \vdots & \ddots & \vdots \\
		0 & 0 & 0 & \dots & 1 \\
		0 & 0 & 0 & \dots & \lambda_i
	\end{pmatrix}.
$$
Матрица перехода $\mathcal{U}$ имеет вид
$$
\mathcal{U} = \left( \mathcal{U}_1, \mathcal{U}_2, \dots, \mathcal{U}_m \right),
$$
где матрицы $\mathcal{U}_i, i= \overline{1,m}$ имеют вид
$$
\mathcal{U}_i = \begin{pmatrix}
1 & 0 & \dots & 0 \\
\lambda_i & 1 & \dots & 0 \\ 
\lambda_i^2 & 2\lambda_i & \dots & 0 \\
\vdots & \vdots & \ddots & \vdots \\
\lambda_i^{n-1} & (n-1)\lambda_i^{n-2} & \dots & \prod_{k=1}^{k_i-1}(n-k)\lambda_i^{n-k_i}
\end{pmatrix}.
$$
}
\textbf{Доказательство.} Пусть $\lambda_i$ - собственное значение матрицы $\mathcal{A}$ кратности~$k_i$. Найдем соответствующие ему собственный и присоединенные векторы. Рассмотрим матрицу вида
$$
\mathcal{A}-\lambda_i I = 
\begin{pmatrix}
	-\lambda_i & 1 & 0 & \dots & 0 \\
	0 & -\lambda_i & 1 & \dots & 0 \\
	\vdots & \vdots & \vdots & \ddots & \vdots \\
	0 & 0 & 0 & \dots & 1 \\
	\alpha_n & \alpha_{n-1} & \alpha_{n-1} & \dots & \alpha_1-\lambda_i
\end{pmatrix},
$$ 
где $I$ -- единичная матрица.\\
Пусть $x_i \in \mathbb{C}^n$ - собственный вектор, отвечающий собственному значению~$\lambda_i$. Тогда справедливо равенство
\begin{equation}\label{Equation1}
\left(\mathcal{A}-\lambda_i I \right) x_i = 0,
\end{equation}
где $x_i = \left( x_1, x_2, \dots, x_n \right)$. Пусть $x_1 = 1$. Тогда равенство \eqref{Equation1} эквивалентно системе:
\begin{equation}\label{SystemForVector}
\begin{cases}
-\lambda_i + x_2 = 0, \\
-\lambda_i x_2 + x_3 = 0, \\
\dots \\
-\lambda_i x_{n-2} + x_{n-1} = 0, \\
\alpha_n + \alpha_{n-1} x_2 + \dots + (\alpha_1 - \lambda_i) x_n = 0.
\end{cases}
\end{equation}
Решив систему \eqref{SystemForVector}, получим
$$
x_i = \left( 1, \lambda_i, \lambda_i ^2, \dots, \lambda_i ^{n-1} \right).
$$
Пусть $k_i \ne 1$. Найдем присоединенные векторы. \\
Пусть $x^j_i \in \mathbb{C}^n,1\le j\le k_i-1$~-~присоединенные векторы, отвечающие собственному значению $\lambda_i$. Первый присоединенный вектор $x^1_i$ есть решение уравнения
\begin{equation}\label{Equation2}
\left(\mathcal{A}-\lambda_i I \right) x^1_i = x_i,
\end{equation}
где $x^1_i =  \left( x_1^{(1)}, x_2^{(1)}, \dots, x_n^{(1)} \right)$ - подлежащий определению присоединенный вектор, $x_i$ - собственный вектор, отвечающий собственному значению $\lambda_i$. Пусть $x_1^{(1)} = 0$. Тогда равенство \eqref{Equation2} эквивалентно системе:
\begin{equation}\label{SystemForVect}
\begin{cases}
x_2^{(1)} = 1, \\
-\lambda_i x_2^{(1)} + x_3^{(1)} = \lambda_i, \\
\dots \\
-\lambda_i x_{n-2}^{(1)} + x_{n-1}^{(1)} = \lambda_i ^{n-2}, \\
\alpha_{n-1} x_2^{(1)} + \alpha_{n-2} x_3^{(1)} \dots + (\alpha_1 - \lambda_i ^{n-1}) x_n^{(1)} = \lambda_i ^{n-1}.
\end{cases}
\end{equation}
Поскольку $P'(\lambda_i)=0$, тогда, решив систему \eqref{SystemForVect}, получим вектор 
$$x^1_i = \left(0, 1, 2 \lambda_i, 3 \lambda_i ^2, \dots, (n-1)\lambda_i ^{n-2} \right), $$ который является присоединенным к собственному вектору $x_i$. \\
Аналогичным образом устанавливается, что векторы 
$$
y^i_p = \left(\underbrace{0, \dots, 0}_{{p}}, p!, (p+1)! \lambda_i, \dots, \prod_{k=1}^p(n-k)\lambda_i ^{n-p-1} \right),
$$
где $1 \le p \le {k_i-1}$, являются присоединенными к вектору $x_i$. \\
Таким образом, доказано что матрица $\mathcal{U}$ составлена из собственных и присоединенных векторов. Поэтому $det(\mathcal{U}) \ne 0$, и, следовательно, матрица $\mathcal{U}$ является матрицей перехода к жордановой форме и имеет вид 
$$
\mathcal{U} = \left( \mathcal{U}_1, \mathcal{U}_2, \dots, \mathcal{U}_m \right),
$$
где матрицы $\mathcal{U}_i, i=1 \dots m$ имеют вид
$$
\mathcal{U}_i = \begin{pmatrix}
1 & 0 & \dots & 0 \\
\lambda_i & 1 & \dots & 0 \\ 
\lambda_i^2 & 2\lambda_i & \dots & 0 \\
\vdots & \vdots & \ddots & \vdots \\
\lambda_i^{n-1} & (n-1)\lambda_i^{n-2} & \dots & \prod_{k=1}^{k_i-1}(n-k)\lambda_i^{n-k_i}
\end{pmatrix}.
$$
Теорема доказана. 

Приведем примеры матрицы $\mathcal{U}$ для двух частных случаев.  \\
\textbf{Пример 1.} \\
Пусть корни характеристического многочлена просты. Тогда матрица $\mathcal{U}$ имеет вид
$$
\mathcal{U} = 
\begin{pmatrix}
	1 & 1 & 1 & \dots & 1 \\
	\lambda_1 & \lambda_2 & \lambda_3 & \dots & \lambda_n \\
	\lambda_1 ^ 2 & \lambda_2 ^2 & \lambda_3 ^2 & \dots & \lambda_n ^2 \\
	\vdots & \vdots & \vdots & \ddots & \vdots \\
	\lambda_1 ^{n-1} & \lambda_2 ^{n-1} &\lambda_3 ^{n-1} & \dots & \lambda_n ^{n-1}
\end{pmatrix}.
$$
Она является матрицей Вандермонда. \\
\textbf{Пример 2.}\\
Пусть характеристический многочлен имеет единственный корень $\lambda_0$. Тогда матрица $\mathcal{U}$ имеет вид
$$
\mathcal{U} = 
\begin{pmatrix}
	1 & 0 & 0 & \dots & 0 \\
	\lambda_0 & 1 & 0 & \dots & 0 \\
	\lambda_0 ^ 2 & 2 \lambda_0 & 2 & \dots & 0 \\
	\vdots & \vdots & \vdots & \ddots & \vdots \\
	\lambda_0 ^{n-1} & (n-1)\lambda_0^{n-2} & (n-1)(n-2)\lambda_0^{n-3} & \dots & (n-1)!
\end{pmatrix}.
$$