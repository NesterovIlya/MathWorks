\section{Теорема}
\textbf{Теорема~1.}
{ \it Пусть $\lambda_1, \ldots , \lambda_m$ собственные значения матрицы $\mathcal{A}$ кратностей $k_1, \ldots , k_m$ соответственно. Тогда жорданова форма для матрицы $\mathcal{A}$ имеет вид:
$$
	\mathcal{J} = \begin{pmatrix}
		\mathcal{J}_1 & 0 & \dots & 0 \\
		0 & \mathcal{J}_2 & \dots & 0 \\
		\vdots & \vdots & \ddots & \vdots \\
		0 & 0 & \dots & \mathcal{J}_m
	\end{pmatrix}, {\text{где }}
	\mathcal{J}_i = \begin{pmatrix}
		\lambda_i & 1 & 0 & \dots & 0 \\
		0 & \lambda_i & 1 & \dots & 0 \\
		\vdots & \vdots & \vdots & \ddots & \vdots \\
		0 & 0 & 0 & \dots & 1 \\
		0 & 0 & 0 & \dots & \lambda_i
	\end{pmatrix}.
$$
Векторами столбцами матрицы перехода являются собственные и присоединенные векторы, где собственные векторы имеют вид: 
$$x^i =(1, \lambda_i, \ldots, \lambda_i^{n-1}), 1 \leqslant i \leqslant m,$$
а присоединенные:
$$y_i^j = (1^{(j)}, \lambda_i^{(j)}, \ldots , (\lambda_i^{n-1})^{(j)}), 1 \leqslant i \leqslant m, 1 \leqslant j \leqslant k_i.$$} 


