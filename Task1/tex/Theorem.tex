\textbf{Теорема~1.}
{ \it Пусть $\lambda_1, \ldots , \lambda_m$ собственные значения матрицы $\mathcal{A}$ кратностей $k_1, \ldots , k_m$ соответственно, где $\sum\limits_{i=1}^m k_i = n$. Тогда жорданова форма для матрицы $\mathcal{A}$ имеет вид:
$$
	\mathcal{J} = \begin{pmatrix}
		\mathcal{J}_1 & 0 & \dots & 0 \\
		0 & \mathcal{J}_2 & \dots & 0 \\
		\vdots & \vdots & \ddots & \vdots \\
		0 & 0 & \dots & \mathcal{J}_m
	\end{pmatrix}, {\text{где }}
	\mathcal{J}_i = \begin{pmatrix}
		\lambda_i & 1 & 0 & \dots & 0 \\
		0 & \lambda_i & 1 & \dots & 0 \\
		\vdots & \vdots & \vdots & \ddots & \vdots \\
		0 & 0 & 0 & \dots & 1 \\
		0 & 0 & 0 & \dots & \lambda_i
	\end{pmatrix}.
$$
Матрица перехода $\mathcal{U}$ имеет вид
$$
\mathcal{U} = \left( \mathcal{U}_1, \mathcal{U}_2, \dots, \mathcal{U}_m \right),
$$
где матрицы $\mathcal{U}_i, i=1 \dots m$ имеют вид
$$
\mathcal{U}_i = \begin{pmatrix}
1 & 0 & \dots & 0 \\
\lambda_i & 1 & \dots & 0 \\ 
\lambda_i^2 & 2\lambda_i & \dots & 0 \\
\vdots & \vdots & \ddots & \vdots \\
\lambda_i^{n-1} & (n-1)\lambda_i^{n-2} & \dots & \prod_{k=1}^{k_i-1}(n-k)\lambda_i^{n-k_i}
\end{pmatrix}.
$$
}
\textbf{Доказательство.} Пусть $\lambda_i$ - собственное значение матрицы $\mathcal{A}$ кратности~$k_i$. Найдем соответствующие ему собственный и присоединенные векторы матрицы.
$$
\mathcal{A}-\lambda_i I = 
\begin{pmatrix}
	-\lambda_i & 1 & 0 & \dots & 0 \\
	0 & -\lambda_i & 1 & \dots & 0 \\
	\vdots & \vdots & \vdots & \ddots & \vdots \\
	0 & 0 & 0 & \dots & 1 \\
	\alpha_n & \alpha_{n-1} & \alpha_{n-1} & \dots & \alpha_1-\lambda_i
\end{pmatrix}.
$$ 
Пусть $x^i \in \mathbb{C}^n$ - собственный вектор, отвечающий собственному значению~$\lambda_i$. Тогда справедливо равенство
\begin{equation}\label{Equation1}
\left(\mathcal{A}-\lambda_i I \right) x^i = 0,
\end{equation}
где $x^i = \left( x_1, x_2, \dots, x_n \right)$. Пусть $x_1 = 1$. Тогда равенство \eqref{Equation1} эквивалентно системе:
\begin{equation}\label{SystemForVector}
\begin{cases}
-\lambda_i + x_2 = 0, \\
-\lambda_i x_2 + x_3 = 0, \\
\dots \\
-\lambda_i x_{n-2} + x_{n-1} = 0, \\
\alpha_n + \alpha_{n-1} x_2 + \dots + (\alpha_1 - \lambda_i) x_n = 0.
\end{cases}
\end{equation}
Решив систему \eqref{SystemForVector}, получим
$
x_2=\lambda_i, x_3=\lambda_i^2, \dots, x_{n-1} = \lambda_i^{n-2},
$
а число $x_n$ однозначно определяется из последнего уравнения системы~\eqref{SystemForVector}~и~оно равно $\lambda_i^{n-1}$. Таким образом, получено, что собственный вектор, отвечающий собственному значению $\lambda_i$, имеет вид
$$
x^i = \left( 1, \lambda_i, \lambda_i ^2, \dots, \lambda_i ^{n-1} \right).
$$
Пусть $k_i \ne 1$. Найдем присоединенные векторы. \\
Пусть $y^i_j \in \mathbb{C}^n,~1~\le~j~\le~k_i-1$~-~присоединенные векторы, отвечающие собственному значению $\lambda_i$. Первый присоединенный вектор $y^i_1$ есть решение уравнения
\begin{equation}\label{Equation2}
\left(\mathcal{A}-\lambda_i I \right) y^i_1 = x^i,
\end{equation}
где $y^i_1 =  \left( y_1^{(1)}, y_2^{(1)}, \dots, y_n^{(1)} \right)$, $x^i$ - собственный вектор, отвечающий собственному значению $\lambda_i$. Пусть $y_1^{(1)} = 0$. Тогда равенство \eqref{Equation2} эквивалентно системе:
\begin{equation}\label{SystemForVect}
\begin{cases}
y_2^{(1)} = 1, \\
-\lambda_i y_2^{(1)} + y_3^{(1)} = \lambda_i, \\
\dots \\
-\lambda_i y_{n-2}^{(1)} + y_{n-1}^{(1)} = \lambda_i ^{n-2}, \\
\alpha_{n-1} y_2^{(1)} + \alpha_{n-2} y_3^{(1)} \dots + (\alpha_1 - \lambda_i ^{n-1}) y_n^{(1)} = \lambda_i ^{n-1}.
\end{cases}
\end{equation}
Решив систему \eqref{SystemForVect}, получим
$$
y^i_1 = \left(0, 1, 2 \lambda_i, 3 \lambda_i ^2, \dots, (n-1)\lambda_i ^{n-2} \right).
$$
Покажем, что найденный $y^i_1$ удовлетворяет системе \eqref{SystemForVect}. Очевидно, что первые~$n-1$~уравнений справеливы, покажем, что справедливо последнее:
\begin{align*}
\alpha_{n-1} + 2 \alpha_{n-2} \lambda_i + \dots + (\alpha_1 - \lambda_i) (n-1) \lambda_i ^{n-2} = \lambda_i ^{n-1}, \\
\alpha_{n-1} + 2 \alpha_{n-2} \lambda_i + \dots + \alpha_1 (n-1)\lambda_i ^{n-2} - n\lambda_i^{n-1} + \lambda_i ^{n-1} = \lambda_i ^{n-1}, \\
\alpha_{n-1} + 2 \alpha_{n-2} \lambda_i + \dots + \alpha_1 (n-1)\lambda_i ^{n-2} - n\lambda_i^{n-1} = 0.
\end{align*}
Ясно, что $\left.P'(\lambda)\right|_{\lambda_i} = 0$, так как $\lambda_i$ - корень многочлена $P$ кратности $k_i \ne 1$.
Если $k_i \geqslant 2$, тогда в силу равенства нулю любой производной до $k_i-1$ порядка многочлена $P$ в точке $\lambda_i$, можно показать, что
$$
y^i_p = \left(\underbrace{0, \dots, 0}_{\text{p}}, p!, (p+1)! \lambda_i, \dots, \prod_{k=1}^p(n-k)\lambda_i ^{n-p-1} \right).
$$
Таким образом, доказано что матрица $\mathcal{U}$ составлена из собственных и присоединенных векторов. Поэтому $det(\mathcal{U}) \ne 0$, и, следовательно, матрица $\mathcal{U}$ является матрицей перехода к жордановой форме и имеет вид 
$$
\mathcal{U} = \left( \mathcal{U}_1, \mathcal{U}_2, \dots, \mathcal{U}_m \right),
$$
где матрицы $\mathcal{U}_i, i=1 \dots m$ имеют вид
$$
\mathcal{U}_i = \begin{pmatrix}
1 & 0 & \dots & 0 \\
\lambda_i & 1 & \dots & 0 \\ 
\lambda_i^2 & 2\lambda_i & \dots & 0 \\
\vdots & \vdots & \ddots & \vdots \\
\lambda_i^{n-1} & (n-1)\lambda_i^{n-2} & \dots & \prod_{k=1}^{k_i-1}(n-k)\lambda_i^{n-k_i}
\end{pmatrix}.
$$
Теорема доказана.

Приведем примеры матрицы $\mathcal{U}$ для двух частных случаев.  \\
Пусть все собственные значения матрицы $\mathcal{A}$ различны. Тогда матрица $\mathcal{U}$ имеет вид
$$
\mathcal{U} = 
\begin{pmatrix}
	1 & 1 & 1 & \dots & 1 \\
	\lambda_1 & \lambda_2 & \lambda_3 & \dots & \lambda_n \\
	\lambda_1 ^ 2 & \lambda_2 ^2 & \lambda_3 ^2 & \dots & \lambda_n ^2 \\
	\vdots & \vdots & \vdots & \ddots & \vdots \\
	\lambda_1 ^{n-1} & \lambda_2 ^{n-1} &\lambda_3 ^{n-1} & \dots & \lambda_n ^{n-1}
\end{pmatrix}.
$$
Она является матрицей Вандермонда. \\
Пусть кратность единственного собственного значения $\lambda_0$ равна $n$. Тогда матрица $\mathcal{U}$ имеет вид
$$
\mathcal{U} = 
\begin{pmatrix}
	1 & 0 & 0 & \dots & 0 \\
	\lambda_0 & 1 & 0 & \dots & 0 \\
	\lambda_0 ^ 2 & 2 \lambda_0 & 2 & \dots & 0 \\
	\vdots & \vdots & \vdots & \ddots & \vdots \\
	\lambda_0 ^{n-1} & (n-1)\lambda_0^{n-2} & (n-1)(n-2)\lambda_0^{n-3} & \dots & (n-1)!
\end{pmatrix}.
$$