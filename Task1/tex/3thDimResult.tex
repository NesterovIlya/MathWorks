\section{Пример}
Для примера рассмотрим частный случай~$n=3$. Тогда матрица~$\mathcal{A}$ имеет вид:
$$
 \mathcal{A} = 
 \begin{pmatrix}
  0 & 1 & 0\\
  0 & 0 & 1 \\
  \alpha_3 & \alpha_2 & \alpha_1 \\
 \end{pmatrix}
$$
Возможны три варианта:
\begin{enumerate}
\item все собственные значения различны;
\item есть собственные значения $\lambda_1, \lambda_2,$ которые имеют кратности ${k_1=2},~{k_2=1}$ соответственно; 
\item все собственные значения одинаковы;        
\end{enumerate}

Рассмотрим первый случай.
Пусть матрица $\mathcal{A}$ имеет собственные значения $\lambda_1, \lambda_2, \lambda_3$ которые имеют кратности~ $k_1=k_2=k_3=1$ соответственно. Оператор $A$ имеет простую структуру. Тогда жорданова форма матрицы~$\mathcal{A}$ имеет вид:
$$
 \mathcal{J} = 
 \begin{pmatrix}
  \lambda_1 & 0 & 0\\
  0 & \lambda_2 & 0 \\
  0 & 0 & \lambda_3 \\
 \end{pmatrix}
$$
Далее запишем матрицу перехода~$\mathcal{U}$ и~обратную к~ней~$\mathcal{U}^{-1}$:
$$
\begin{aligned}
 \mathcal{U} &= 
 \begin{pmatrix}
  	1 & 1 & 1\\
  	\lambda_1 & \lambda_2 & \lambda_3 \\
  	\lambda_1^2 & \lambda_2^2 & \lambda_3^2 \\
 \end{pmatrix}, \\
 \mathcal{U}^{-1} &= \frac{1}{\Delta} 
 \begin{pmatrix}
  	\lambda_2\lambda_3(\lambda_3-\lambda_2) & \lambda_2^2-\lambda_3^2 & \lambda_3-\lambda_2\\
  	\lambda_1\lambda_3(\lambda_1-\lambda_3) & \lambda_3^2-\lambda_1^2 & \lambda_1-\lambda_3\\
  	\lambda_1\lambda_2(\lambda_2-\lambda_1) & \lambda_1^2-\lambda_2^2 & \lambda_2-\lambda_1\\
 \end{pmatrix},
\end{aligned}
$$
где $\Delta = \det \mathcal{U} = (\lambda_3-\lambda_1)(\lambda_3-\lambda_2)(\lambda_2-\lambda_1).$ \\
Теперь запишем проекторы оператора~$A$:

$$
	\begin{aligned}
		\mathcal{A} &= \mathcal{U}\mathcal{J}\mathcal{U}^{-1}, \\
		\mathcal{P}_i &= \mathcal{U}\mathcal{P}_i'\mathcal{U}^{-1}, \\
		\mathcal{P}_i' &= \left(p_{jk}'\right), \\
		p_{jk}' &= 
		\begin{cases}
			1, & \text{если $j=k=i$;} \\
			0, & \text{в противном случае.}
		\end{cases}
	\end{aligned}
$$
где $i,j,k \in \{1,2,3\}, \text{и}\  \mathcal{P}_i'$~--- проекторы жордановой матрицы~$\mathcal{J}$ 
оператора~$A$. Тогда по теореме о спектральном разложении оператора простой структуры \cite{baskakov}:
$$
	\mathcal{A} = \lambda_1\mathcal{P}_1 + \lambda_2\mathcal{P}_2 + \lambda_3\mathcal{P}_3 
$$


Рассмотрим второй случай.		
Пусть матрица~$\mathcal{A}$ имеет собственные значения~$\lambda_1,~\lambda_2$ 
которые имеют кратности~$k_1=2,~k_2=1$ соответственно. Тогда жорданова форма матрицы~$\mathcal{A}$ имеет вид:
$$
 \mathcal{J} = 
 \begin{pmatrix}
  \lambda_1 & 1 & 0\\
  0 & \lambda_1 & 0 \\
  0 & 0 & \lambda_2 \\
 \end{pmatrix}
$$
Запишем матрицу перехода~$\mathcal{U}$ и~обратную к~ней~$\mathcal{U}^{-1}$:
$$
\begin{aligned}
 \mathcal{U} &= 
 \begin{pmatrix}
  	1 & 0 & 1\\
  	\lambda_1 & 1 & \lambda_2 \\
  	\lambda_1^2 & 2\lambda_1 & \lambda_2^2 \\
 \end{pmatrix}, \\
 \mathcal{U}^{-1} &= \frac{1}{\Delta} 
 \begin{pmatrix}
  	\lambda_2^2-2\lambda_1\lambda_2 & 2\lambda_1 & -1  \\
  	\lambda_1^2\lambda_2 - \lambda_1\lambda_2^2 & \lambda_2^2 - \lambda_1^2 & \lambda_1 - \lambda_2  \\
  	\lambda_1^2 & -2\lambda_1 & 1  \\
 \end{pmatrix},
\end{aligned}
$$
где $\Delta = \det \mathcal{U} = (\lambda_1-\lambda_2)^2$. \\
Для спектрального разложения найдем проекторы и нильпотентную часть оператора~$\ A$:
$$
	\begin{aligned}
		\mathcal{A} &= \mathcal{U}\mathcal{J}\mathcal{U}^{-1}, \\
		\mathcal{P}_i &= \mathcal{U}\mathcal{P}_i'\mathcal{U}^{-1}, \\
		\mathcal{P}_1' = 
		\begin{pmatrix}
  			1 & 0 & 0\\
  			0 & 1 & 0 \\
  			0 & 0 & 0 \\
 		\end{pmatrix}&,\  
 		\mathcal{P}_2' =
		\begin{pmatrix}
  			0 & 0 & 0\\
  			0 & 0 & 0 \\
  			0 & 0 & 1 \\
 		\end{pmatrix},\\ 
 		\mathcal{Q} &= \mathcal{U}\mathcal{Q}'\mathcal{U}^{-1}, \\
 		\mathcal{Q}' &= 
		\begin{pmatrix}
  			0 & 1 & 0\\
  			0 & 0 & 0 \\
  			0 & 0 & 0 \\
 		\end{pmatrix}
	\end{aligned}
$$
где $i \in \{1,2\}, \text{и}\  \mathcal{P}_i', \mathcal{Q}'$~--- проекторы жордановой матрицы~$\mathcal{J}$ 
оператора~$A$ и её нильпотентная часть соответственно. Тогда по теореме о спектральном разложении линейного оператора \cite{baskakov}:
$$
	\mathcal{A} = \lambda_1\mathcal{P}_1 + \lambda_2\mathcal{P}_2 + \mathcal{Q} 
$$

И наконец последний случай.		
Пусть матрица~$\mathcal{A}$ имеет одно собственное значение~$\lambda$ 
кратности~$k=3$. Тогда жорданова форма матрицы~$\mathcal{A}$ имеет вид:
$$
 \mathcal{J} = 
 \begin{pmatrix}
  \lambda & 1 & 0\\
  0 & \lambda & 1 \\
  0 & 0 & \lambda \\
 \end{pmatrix}
$$
Запишем матрицу перехода~$\mathcal{U}$ и~обратную к~ней~$\mathcal{U}^{-1}$:
$$
\begin{aligned}
 \mathcal{U} &= 
 \begin{pmatrix}
  	1 & 0 & 0\\
  	\lambda & 1 & 0 \\
  	\lambda^2 & 2\lambda & 2 \\
 \end{pmatrix}, \\
 \mathcal{U}^{-1} &= \frac{1}{\Delta} 
 \begin{pmatrix}
  	1 & 0 & 0  \\
  	-\lambda & 1 & 0  \\
  	\frac{1}{2}\lambda^2 & -\lambda & \frac{1}{2} \\
 \end{pmatrix},
\end{aligned}
$$
где $\Delta = \det \mathcal{U} = 2$. \\
Спектральное разложение для этого случая очевидно.