В качестве примера рассмотрим частный случай~$n=3$. Тогда матрица~$\mathcal{A}$ имеет вид:
$$
 \mathcal{A} = 
 \begin{pmatrix}
  0 & 1 & 0\\
  0 & 0 & 1 \\
  \alpha_3 & \alpha_2 & \alpha_1 \\
 \end{pmatrix}
$$
Возможны три варианта: \\
%\begin{enumerate}
1) все собственные значения различны; \\
2) собственные значения $\lambda_1, \lambda_2$ имеют кратности ${k_1=2},~{k_2=1}$ (либо наоборот); \\
3) матрица $\mathcal{A}$ имеет одно собственное значение $\lambda_0$;        
%\end{enumerate}

\textbf{Рассмотрим первый случай.} 
Пусть матрица $\mathcal{A}$ имеет собственные значения $\lambda_1, \lambda_2, \lambda_3$ кратностей~ $k_1=k_2=k_3=1$ соответственно. Оператор $A$ имеет простую структуру. Тогда жорданова форма матрицы~$\mathcal{A}$ имеет вид:
$$
 \mathcal{J} = 
 \begin{pmatrix}
  \lambda_1 & 0 & 0\\
  0 & \lambda_2 & 0 \\
  0 & 0 & \lambda_3 \\
 \end{pmatrix}
$$
Далее запишем матрицу перехода~$\mathcal{U}$ и~обратную к~ней~$\mathcal{U}^{-1}$:
$$
\begin{aligned}
 \mathcal{U} &= 
 \begin{pmatrix}
  	1 & 1 & 1\\
  	\lambda_1 & \lambda_2 & \lambda_3 \\
  	\lambda_1^2 & \lambda_2^2 & \lambda_3^2 \\
 \end{pmatrix}, \\
 \mathcal{U}^{-1} &= \frac{1}{\Delta} 
 \begin{pmatrix}
  	\lambda_2\lambda_3(\lambda_3-\lambda_2) & \lambda_2^2-\lambda_3^2 & \lambda_3-\lambda_2\\
  	\lambda_1\lambda_3(\lambda_1-\lambda_3) & \lambda_3^2-\lambda_1^2 & \lambda_1-\lambda_3\\
  	\lambda_1\lambda_2(\lambda_2-\lambda_1) & \lambda_1^2-\lambda_2^2 & \lambda_2-\lambda_1\\
 \end{pmatrix},
\end{aligned}
$$
где $\Delta = \det \mathcal{U} = (\lambda_3-\lambda_1)(\lambda_3-\lambda_2)(\lambda_2-\lambda_1).$ \\
Рассмотрим проекторы $\mathcal{P}_i'$ жордановой матрицы~$\mathcal{J}$ оператора~$A$, которые имеют вид:
$$
\begin{aligned}
\mathcal{P}_i' &= \left(p_{jk}'\right), \\
		p_{jk}' &= 
		\begin{cases}
			1, & \text{если $j=k=i$;} \\
			0, & \text{в противном случае.}
		\end{cases}
\end{aligned}
$$
где $i,j,k \in \{1,2,3\}$.\\
Теперь запишем проекторы оператора~$A$:
$$
	\begin{aligned}
		\mathcal{P}_i &= \mathcal{U}\mathcal{P}_i'\mathcal{U}^{-1}, 
	\end{aligned}
$$
где $i \in \{1,2,3\}$. \\
Найдем проектор $\mathcal{P}_1$
$$
\mathcal{P}_1 = 
\frac{1}{\Delta}
\begin{pmatrix}
  	1 & 1 & 1\\
  	\lambda_1 & \lambda_2 & \lambda_3 \\
  	\lambda_1^2 & \lambda_2^2 & \lambda_3^2 \\
 \end{pmatrix}
\begin{pmatrix}
  	1 & 0 & 0\\
  	0 & 0 & 0\\
  	0 & 0 & 0 \\
 \end{pmatrix}
\begin{pmatrix}
  	\lambda_2\lambda_3(\lambda_3-\lambda_2) & \lambda_2^2-\lambda_3^2 & \lambda_3-\lambda_2\\
  	\lambda_1\lambda_3(\lambda_1-\lambda_3) & \lambda_3^2-\lambda_1^2 & \lambda_1-\lambda_3\\
  	\lambda_1\lambda_2(\lambda_2-\lambda_1) & \lambda_1^2-\lambda_2^2 & \lambda_2-\lambda_1\\
 \end{pmatrix},
$$
$$
\mathcal{P}_1 = 
\frac{1}{\Delta}
\begin{pmatrix}
  	1 & 0 & 0\\
  	\lambda_1 & 0 & 0 \\
  	\lambda_1^2 & 0 & 0 \\
 \end{pmatrix}
\begin{pmatrix}
  	\lambda_2\lambda_3(\lambda_3-\lambda_2) & \lambda_2^2-\lambda_3^2 & \lambda_3-\lambda_2\\
  	\lambda_1\lambda_3(\lambda_1-\lambda_3) & \lambda_3^2-\lambda_1^2 & \lambda_1-\lambda_3\\
  	\lambda_1\lambda_2(\lambda_2-\lambda_1) & \lambda_1^2-\lambda_2^2 & \lambda_2-\lambda_1\\
 \end{pmatrix},
$$
$$
\mathcal{P}_1 = 
\frac{1}{\Delta}
\begin{pmatrix}
  	\lambda_2\lambda_3(\lambda_3-\lambda_2) & \lambda_2^2-\lambda_3^2 & \lambda_3-\lambda_2\\
  	\lambda_1\lambda_2\lambda_3(\lambda_3-\lambda_2) & \lambda_1\left(\lambda_2^2-\lambda_3^2\right) & \lambda_1\left(\lambda_3-\lambda_2\right)\\
  	\lambda_1^2\lambda_2\lambda_3(\lambda_3-\lambda_2) & \lambda_1^2\left(\lambda_2^2-\lambda_3^2\right) & \lambda_1^2\left(\lambda_3-\lambda_2\right)\\
 \end{pmatrix},
$$
где $\Delta = (\lambda_3-\lambda_1)(\lambda_3-\lambda_2)(\lambda_2-\lambda_1).$ \\
Найдем проектор $\mathcal{P}_2$
$$
\mathcal{P}_2 = 
\frac{1}{\Delta}
\begin{pmatrix}
  	1 & 1 & 1\\
  	\lambda_1 & \lambda_2 & \lambda_3 \\
  	\lambda_1^2 & \lambda_2^2 & \lambda_3^2 \\
 \end{pmatrix}
\begin{pmatrix}
  	0 & 0 & 0\\
  	0 & 1 & 0\\
  	0 & 0 & 0 \\
 \end{pmatrix}
\begin{pmatrix}
  	\lambda_2\lambda_3(\lambda_3-\lambda_2) & \lambda_2^2-\lambda_3^2 & \lambda_3-\lambda_2\\
  	\lambda_1\lambda_3(\lambda_1-\lambda_3) & \lambda_3^2-\lambda_1^2 & \lambda_1-\lambda_3\\
  	\lambda_1\lambda_2(\lambda_2-\lambda_1) & \lambda_1^2-\lambda_2^2 & \lambda_2-\lambda_1\\
 \end{pmatrix},
$$
$$
\mathcal{P}_2 = 
\frac{1}{\Delta}
\begin{pmatrix}
  	0 & 1 & 0\\
  	0 & \lambda_2 & 0 \\
  	0 & \lambda_2^2 & 0 \\
 \end{pmatrix}
\begin{pmatrix}
  	\lambda_2\lambda_3(\lambda_3-\lambda_2) & \lambda_2^2-\lambda_3^2 & \lambda_3-\lambda_2\\
  	\lambda_1\lambda_3(\lambda_1-\lambda_3) & \lambda_3^2-\lambda_1^2 & \lambda_1-\lambda_3\\
  	\lambda_1\lambda_2(\lambda_2-\lambda_1) & \lambda_1^2-\lambda_2^2 & \lambda_2-\lambda_1\\
 \end{pmatrix},
$$
$$
\mathcal{P}_2 = 
\frac{1}{\Delta}
\begin{pmatrix}
  	\lambda_1\lambda_3(\lambda_1-\lambda_3) & \lambda_3^2-\lambda_1^2 & \lambda_1-\lambda_3\\
  	\lambda_1\lambda_2\lambda_3(\lambda_1-\lambda_3) & \lambda_2(\lambda_3^2-\lambda_1^2) & \lambda_2(\lambda_1-\lambda_3)\\
  	\lambda_1\lambda_2^2\lambda_3(\lambda_1-\lambda_3) & \lambda_2^2(\lambda_3^2-\lambda_1^2) & \lambda_2^2(\lambda_1-\lambda_3)\\
 \end{pmatrix},
$$
где $\Delta = (\lambda_3-\lambda_1)(\lambda_3-\lambda_2)(\lambda_2-\lambda_1).$ \\
Найдем проектор $\mathcal{P}_3$
$$
\mathcal{P}_3 = 
\frac{1}{\Delta}
\begin{pmatrix}
  	1 & 1 & 1\\
  	\lambda_1 & \lambda_2 & \lambda_3 \\
  	\lambda_1^2 & \lambda_2^2 & \lambda_3^2 \\
 \end{pmatrix}
\begin{pmatrix}
  	0 & 0 & 0\\
  	0 & 0 & 0\\
  	0 & 0 & 1 \\
 \end{pmatrix}
\begin{pmatrix}
  	\lambda_2\lambda_3(\lambda_3-\lambda_2) & \lambda_2^2-\lambda_3^2 & \lambda_3-\lambda_2\\
  	\lambda_1\lambda_3(\lambda_1-\lambda_3) & \lambda_3^2-\lambda_1^2 & \lambda_1-\lambda_3\\
  	\lambda_1\lambda_2(\lambda_2-\lambda_1) & \lambda_1^2-\lambda_2^2 & \lambda_2-\lambda_1\\
 \end{pmatrix},
$$
$$
\mathcal{P}_3 = 
\frac{1}{\Delta}
\begin{pmatrix}
  	0 & 0 & 1\\
  	0 & 0 & \lambda_3 \\
  	0 & 0 & \lambda_3^2 \\
 \end{pmatrix}
\begin{pmatrix}
  	\lambda_2\lambda_3(\lambda_3-\lambda_2) & \lambda_2^2-\lambda_3^2 & \lambda_3-\lambda_2\\
  	\lambda_1\lambda_3(\lambda_1-\lambda_3) & \lambda_3^2-\lambda_1^2 & \lambda_1-\lambda_3\\
  	\lambda_1\lambda_2(\lambda_2-\lambda_1) & \lambda_1^2-\lambda_2^2 & \lambda_2-\lambda_1\\
 \end{pmatrix},
$$
$$
\mathcal{P}_3 = 
\frac{1}{\Delta}
\begin{pmatrix}
  	\lambda_1\lambda_2(\lambda_2-\lambda_1) & \lambda_1^2-\lambda_2^2 & \lambda_2-\lambda_1\\
  	\lambda_1\lambda_2\lambda_3(\lambda_2-\lambda_1) & \lambda_3(\lambda_1^2-\lambda_2^2) & \lambda_3(\lambda_2-\lambda_1)\\
  	\lambda_1\lambda_2\lambda_3^2(\lambda_2-\lambda_1) & \lambda_3^2(\lambda_1^2-\lambda_2^2) & \lambda_3^2(\lambda_2-\lambda_1)\\
 \end{pmatrix},
$$
где $\Delta = (\lambda_3-\lambda_1)(\lambda_3-\lambda_2)(\lambda_2-\lambda_1).$

Тогда по теореме о спектральном разложении оператора простой структуры [2]:
$$
	\mathcal{A} = \lambda_1\mathcal{P}_1 + \lambda_2\mathcal{P}_2 + \lambda_3\mathcal{P}_3 
$$

\textbf{Рассмотрим второй случай.} 		
Пусть матрица~$\mathcal{A}$ имеет собственные значения~$\lambda_1,~\lambda_2$ 
которые имеют кратности~$k_1=2,~k_2=1$ соответственно. Тогда жорданова форма матрицы~$\mathcal{A}$ имеет вид:
$$
 \mathcal{J} = 
 \begin{pmatrix}
  \lambda_1 & 1 & 0\\
  0 & \lambda_1 & 0 \\
  0 & 0 & \lambda_2 \\
 \end{pmatrix}
$$
Запишем матрицу перехода~$\mathcal{U}$ и~обратную к~ней~$\mathcal{U}^{-1}$:
$$
\begin{aligned}
 \mathcal{U} &= 
 \begin{pmatrix}
  	1 & 0 & 1\\
  	\lambda_1 & 1 & \lambda_2 \\
  	\lambda_1^2 & 2\lambda_1 & \lambda_2^2 \\
 \end{pmatrix}, \\
 \mathcal{U}^{-1} &= \frac{1}{\Delta} 
 \begin{pmatrix}
  	\lambda_2^2-2\lambda_1\lambda_2 & 2\lambda_1 & -1  \\
  	\lambda_1^2\lambda_2 - \lambda_1\lambda_2^2 & \lambda_2^2 - \lambda_1^2 & \lambda_1 - \lambda_2  \\
  	\lambda_1^2 & -2\lambda_1 & 1  \\
 \end{pmatrix},
\end{aligned}
$$
где $\Delta = \det \mathcal{U} = (\lambda_1-\lambda_2)^2$. \\
Рассмотрим проекторы $\mathcal{P}_i', i \in \{1,2\}$ жордановой матрицы~$\mathcal{J}$ оператора~$A$ и её нильпотентную часть $\mathcal{Q}'$.
$$
\begin{aligned}
\mathcal{P}_1' = 
		\begin{pmatrix}
  			1 & 0 & 0\\
  			0 & 1 & 0 \\
  			0 & 0 & 0 \\
 		\end{pmatrix}&,\  
 		\mathcal{P}_2' =
		\begin{pmatrix}
  			0 & 0 & 0\\
  			0 & 0 & 0 \\
  			0 & 0 & 1 \\
 		\end{pmatrix},\\
 		 \mathcal{Q}' &= 
		\begin{pmatrix}
  			0 & 1 & 0\\
  			0 & 0 & 0 \\
  			0 & 0 & 0 \\
 		\end{pmatrix}
\end{aligned}
$$
Теперь запишем проекторы оператора $A$ и нильпотентную часть жордановой матрицы:
$$
	\begin{aligned}
		\mathcal{P}_i &= \mathcal{U}\mathcal{P}_i'\mathcal{U}^{-1}, \\ 
 		\mathcal{Q} &= \mathcal{U}\mathcal{Q}'\mathcal{U}^{-1}, 
	\end{aligned}
$$
где $i \in \{1,2\}$. \\
Найдем проектор $\mathcal{P}_1$
$$
\mathcal{P}_1 = 
\begin{pmatrix}
  	1 & 0 & 1\\
  	\lambda_1 & 1 & \lambda_2 \\
  	\lambda_1^2 & 2\lambda_1 & \lambda_2^2 \\
\end{pmatrix}
 \begin{pmatrix}
  	1 & 0 & 0\\
  	0 & 1 & 0 \\
  	0 & 0 & 0 \\
\end{pmatrix}
\begin{pmatrix}
  	\lambda_2^2-2\lambda_1\lambda_2 & 2\lambda_1 & -1  \\
  	\lambda_1^2\lambda_2 - \lambda_1\lambda_2^2 & \lambda_2^2 - \lambda_1^2 & \lambda_1 - \lambda_2  \\
  	\lambda_1^2 & -2\lambda_1 & 1  \\
 \end{pmatrix} 		
$$
$$
\mathcal{P}_1 = 
\begin{pmatrix}
  	1 & 0 & 0\\
  	\lambda_1 & 1 & 0 \\
  	\lambda_1^2 & 2\lambda_1 & 0 \\
\end{pmatrix}
\begin{pmatrix}
  	\lambda_2^2-2\lambda_1\lambda_2 & 2\lambda_1 & -1  \\
  	\lambda_1^2\lambda_2 - \lambda_1\lambda_2^2 & \lambda_2^2 - \lambda_1^2 & \lambda_1 - \lambda_2  \\
  	\lambda_1^2 & -2\lambda_1 & 1  \\
 \end{pmatrix} 		
$$
$$
\mathcal{P}_1 = 
\begin{pmatrix}
  	\lambda_2^2-2\lambda_1\lambda_2 & 2\lambda_1 & -1  \\
  	-\lambda_1^2\lambda_2 & \lambda_1^2+\lambda_2^2  & - \lambda_2  \\
  	-\lambda_1^2\lambda_2^2 & 2\lambda_1\lambda_2^2 & \lambda_1^2-1\lambda_1\lambda_2  \\
 \end{pmatrix} 	
$$
Найдем проектор $\mathcal{P}_2$
$$
\mathcal{P}_2 = 
\begin{pmatrix}
  	1 & 0 & 1\\
  	\lambda_1 & 1 & \lambda_2 \\
  	\lambda_1^2 & 2\lambda_1 & \lambda_2^2 \\
\end{pmatrix}
 \begin{pmatrix}
  	0 & 0 & 0\\
  	0 & 0 & 0 \\
  	0 & 0 & 1 \\
\end{pmatrix}
\begin{pmatrix}
  	\lambda_2^2-2\lambda_1\lambda_2 & 2\lambda_1 & -1  \\
  	\lambda_1^2\lambda_2 - \lambda_1\lambda_2^2 & \lambda_2^2 - \lambda_1^2 & \lambda_1 - \lambda_2  \\
  	\lambda_1^2 & -2\lambda_1 & 1  \\
 \end{pmatrix} 	
$$
$$
\mathcal{P}_2 = 
\begin{pmatrix}
  	0 & 0 & 1\\
  	0 & 0 & \lambda_2 \\
  	0 & 0 & \lambda_2^2 \\
\end{pmatrix}
\begin{pmatrix}
  	\lambda_2^2-2\lambda_1\lambda_2 & 2\lambda_1 & -1  \\
  	\lambda_1^2\lambda_2 - \lambda_1\lambda_2^2 & \lambda_2^2 - \lambda_1^2 & \lambda_1 - \lambda_2  \\
  	\lambda_1^2 & -2\lambda_1 & 1  \\
 \end{pmatrix} 
$$
$$
\mathcal{P}_2 = 
\begin{pmatrix}
  	\lambda_1^2 & -2\lambda_1 & 1  \\
  	\lambda_1^2\lambda_2 & -2\lambda_1\lambda_2 & \lambda_2  \\
  	\lambda_1^2\lambda_2^2 & -2\lambda_1\lambda_2^2 & \lambda_2^2  \\
 \end{pmatrix} 
$$
Найдем нильпотентную часть жордановой матрицы $\mathcal{Q}$
$$
\mathcal{Q}=
\begin{pmatrix}
  	1 & 0 & 1\\
  	\lambda_1 & 1 & \lambda_2 \\
  	\lambda_1^2 & 2\lambda_1 & \lambda_2^2 \\
\end{pmatrix}
 \begin{pmatrix}
  	0 & 1 & 0\\
  	0 & 0 & 0 \\
  	0 & 0 & 0 \\
\end{pmatrix}
\begin{pmatrix}
  	\lambda_2^2-2\lambda_1\lambda_2 & 2\lambda_1 & -1  \\
  	\lambda_1^2\lambda_2 - \lambda_1\lambda_2^2 & \lambda_2^2 - \lambda_1^2 & \lambda_1 - \lambda_2  \\
  	\lambda_1^2 & -2\lambda_1 & 1  \\
 \end{pmatrix}
$$
$$
\mathcal{Q}=
 \begin{pmatrix}
  	0 & 1 & 0\\
  	0 & \lambda_1 & 0 \\
  	0 & \lambda_1^2 & 0 \\
\end{pmatrix}
\begin{pmatrix}
  	\lambda_2^2-2\lambda_1\lambda_2 & 2\lambda_1 & -1  \\
  	\lambda_1^2\lambda_2 - \lambda_1\lambda_2^2 & \lambda_2^2 - \lambda_1^2 & \lambda_1 - \lambda_2  \\
  	\lambda_1^2 & -2\lambda_1 & 1  \\
 \end{pmatrix}
$$
$$
\mathcal{Q}=
\begin{pmatrix}
  	\lambda_1^2\lambda_2 - \lambda_1\lambda_2^2 & \lambda_2^2 - \lambda_1^2 & \lambda_1 - \lambda_2  \\
  	\lambda_1(\lambda_1^2\lambda_2 - \lambda_1\lambda_2^2) & \lambda_1(\lambda_2^2 - \lambda_1^2) & \lambda_1(\lambda_1 - \lambda_2)  \\
  	\lambda_1^2(\lambda_1^2\lambda_2 - \lambda_1\lambda_2^2) & \lambda_1^2(\lambda_2^2 - \lambda_1^2) & \lambda_1^2(\lambda_1 - \lambda_2)  \\
 \end{pmatrix}
$$

Тогда по теореме о спектральном разложении линейного оператора [2]:
$$
	\mathcal{A} = \lambda_1\mathcal{P}_1 + \lambda_2\mathcal{P}_2 + \mathcal{Q}.
$$

\textbf{Рассмотрим третий случай.}	
Пусть матрица~$\mathcal{A}$ имеет одно собственное значение~$\lambda$ 
кратности~$k=3$. Тогда жорданова форма матрицы~$\mathcal{A}$ имеет вид:
$$
 \mathcal{J} = 
 \begin{pmatrix}
  \lambda & 1 & 0\\
  0 & \lambda & 1 \\
  0 & 0 & \lambda \\
 \end{pmatrix}
$$
Запишем матрицу перехода~$\mathcal{U}$ и~обратную к~ней~$\mathcal{U}^{-1}$:
$$
\begin{aligned}
 \mathcal{U} &= 
 \begin{pmatrix}
  	1 & 0 & 0\\
  	\lambda & 1 & 0 \\
  	\lambda^2 & 2\lambda & 2 \\
 \end{pmatrix}, \\
 \mathcal{U}^{-1} &= \frac{1}{2} 
 \begin{pmatrix}
  	1 & 0 & 0  \\
  	-\lambda & 1 & 0  \\
  	\frac{1}{2}\lambda^2 & -\lambda & \frac{1}{2} \\
 \end{pmatrix}.
\end{aligned}
$$
Спектральное разложение для этого случая очевидно.