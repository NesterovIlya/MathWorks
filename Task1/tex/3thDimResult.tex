Для примера рассмотрим частный случай $n=3.$ Тогда матрица $\mathcal{A}$ имеет вид:
$$
 \mathcal{A} = 
 \begin{pmatrix}
  0 & 1 & 0\\
  0 & 0 & 1 \\
  \alpha_3 & \alpha_2 & \alpha_1 \\
 \end{pmatrix}
$$
Возможны три варианта:
\begin{enumerate} %[\quad 1)]
\item все собственные значения различны;
\item есть собственные значения $\lambda_1, \lambda_2,$ которые имеют кратности $k_1=2, k_2=1$ соответственно; 
\item все собственные значения одинаковы;        
\end{enumerate}

Пусть матрица $\mathcal{A}$ имеет собственные значения $\lambda_1, \lambda_2, \lambda_3$ которые имеют кратности $k_1=k_2=k_3=1$ соответственно. Тогда Жорданова форма матрицы $\mathcal{A}$ имеет вид:
$$
 \mathcal{J} = 
 \begin{pmatrix}
  \lambda_1 & 0 & 0\\
  0 & \lambda_2 & 0 \\
  0 & 0 & \lambda_3 \\
 \end{pmatrix}
$$
Далее запишем матрицу перехода $\mathcal{U}$ и обратную к ней $\mathcal{U}^{-1}.$
$$
 \mathcal{U} = 
 \begin{pmatrix}
  1 & 1 & 1\\
  \lambda_1 & \lambda_2 & \lambda_3 \\
  \lambda_1^2 & \lambda_2^2 & \lambda_3^2 \\
 \end{pmatrix}
\  \mathcal{U}^{-1} = \frac{1}{\Delta} 
 \begin{pmatrix}
  \lambda_2\lambda_3(\lambda_3-\lambda_2) & \lambda_2^2-\lambda_3^2 & \lambda_3-\lambda_2\\
  \lambda_1\lambda_3(\lambda_1-\lambda_3) & \lambda_3^2-\lambda_1^2 & \lambda_1-\lambda_3\\
  \lambda_1\lambda_1(\lambda_2-\lambda_1) & \lambda_2^2-\lambda_1^2 & \lambda_2-\lambda_1\\
 \end{pmatrix},
$$
где $\Delta = (\lambda_3-\lambda_1)(\lambda_3-\lambda_2)(\lambda_2-\lambda_1).$