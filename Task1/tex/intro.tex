
\cite{baskakov}
Пусть $X \subset \mathbb{C}^n$ --- комплексное евклидово пространство размерности $n$. 
Задан вектор $x_0$ и известен набор векторов $x_1,x_2, \dots x_m$, обладающих свойствами:
\begin{enumerate}
	\item векторы $x_0,x_1, \dots x_{m-1}$ линейно независимы;
	\item $x_k = A^k x_0, k= \overline{1,m}$;
	\item $x_m = \alpha_0 x_0 + \alpha_1 x_1 + \dots \alpha_{n-1} x_{n-1}$,
\end{enumerate}
где $A \subset L(X)$ --- неизвестный  линейный оператор и $m \leqslant n$. Положим $m=n$. 
Тогда векторы $x_0 \dots x_{n-1}$ образуют базис в пространстве $X$. \\

Построим матрицу оператора $A$ в этом базисе:
\[
	\begin{pmatrix}
		0 & 1 & 0 & \dots & 0 \\
		0 & 0 & 1 & \dots & 0 \\
		\vdots & \vdots & \vdots & \ddots & \vdots \\
		0 & 0 & 0 & \dots & 1 \\
		\alpha_0 & \alpha_1 & \alpha_2 & \dots & \alpha_n
	\end{pmatrix}
\]
Рассмотрим характеристический многочлен оператора $A$:
\[
	p(\lambda) = \lambda^n+a_1 \lambda^{n-1} + \dots + a_n
\]