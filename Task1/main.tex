\documentclass[10pt,a4paper,oneside]{article}
\usepackage[T2A]{fontenc}
\usepackage[utf8]{inputenc}
\usepackage[russian]{babel}
\usepackage{mathtools}
\usepackage{amssymb}
\usepackage{setspace}
\usepackage{amsthm}


\begin{document}
 
\begin{center}
    \setstretch{1.5}
    \textbf{\LARGE ЖОРДАНОВА ФОРМА ДЛЯ СОПРОВОЖДАЮЩИХ МАТРИЦ} \\[1em]
    \textbf{И.\,Н.\,Нестеров,\,С.\,В.\,Клочков,\,А.\,С.\,Чурсанова} \\[2em]
\end{center}

Рассмотрим линейный оператор $\mathbb{A} \in \spaceend\mathfrak{X},$
где $\mathfrak{X}$ банахово пространство, заданный операторной матрицей, т. \! е.
$$
\mathbb{A} = \begin{pmatrix}
		A & C \\
		D & B
	\end{pmatrix}.
$$
Пусть пространство $\mathfrak{X}$ представимо в виде прямой суммы подпространств: $\mathfrak{X} = \mathfrak{X}_1 \oplus \mathfrak{X}_2.$ Тогда
\begin{align*}
A \colon \mathfrak{X}_1 \to \mathfrak{X}_1 \\
B \colon \mathfrak{X}_2 \to \mathfrak{X}_2 \\
C \colon \mathfrak{X}_2 \to \mathfrak{X}_1 \\
D \colon \mathfrak{X}_1 \to \mathfrak{X}_2
\end{align*}
Представим оператор $\mathbb{A}$ в виде $\mathbb{A} = \mathcal{A} - \mathcal{B},$ где оператор $\mathcal{A} \in \spaceend(\mathfrak{X}_1 \times \mathfrak{X}_2)$ задается матрицей 
$\begin{pmatrix}
		A & 0 \\
		0 & B
\end{pmatrix}, $ а оператор $\mathcal{B} \in \spaceend(\mathfrak{X}_1 \times \mathfrak{X}_2)$ -- матрицей
$\begin{pmatrix}
		0 & -C \\
		-D & 0
\end{pmatrix}.$
Всюду далее считаем что выполняется условие:
$$
\sigma(A) \cap \sigma(B) = {\varnothing}.
$$

Символом $\mathcal{U}$ обозначим пространство $\spaceend(\mathfrak{X}_1 \times \mathfrak{X}_2)$ Рассмотрим канонические проекторы
$$
P_1x = (x_1, 0), P_2x = (0, x_2), x = (x_1, x_2) \in \mathfrak{X}_1 \times \mathfrak{X}_2.
$$
Для любого оператора $X \in \mathcal{U}$ рассмотрим операторы $P_iXP_j \in \mathcal{U}, i,j \in {1,2}.$ Таким образом, оператор $X$ задается матрицей:
$$
\mathcal{X} = \begin{pmatrix}
		X_{11} & X_{12} \\
		X_{21} & X_{22}
	\end{pmatrix},
$$
где оператор $X_{ij}$ -- сужение оператора $P_iXP_j$ на подпространство $\mathfrak{X}_i$ с областью значений $\mathfrak{X}_j, i,j \in {1,2}.$

В соответствии с заданным разложением пространства $\mathfrak{X}$ будем рассматривать два трансформатора: $\mathcal{J} \in \spaceend\mathcal{U}, \Gamma \in \spaceend\mathcal{U}$, таких что:
\begin{enumerate}
	\item Для любого $X \in \mathcal{U}$ оператор $\mathcal{J}X$ определяется следующим образом: $\mathcal{J}X = P_1XP_1 + P_2XP_2,$ а матрица оператора $\mathcal{J}X$ имеет вид:
	$$
	\mathcal{J}X = \begin{pmatrix}
		X_{11} & 0 \\
		0 & X_{22}
	\end{pmatrix};
	$$
	\item Пусть $\Gamma X = Y,$ тогда оператор $Y$ определяется как решение уравнения:
	\begin{equation}\label{eq:gamma_x_rule}
		\mathcal{A}Y - Y\mathcal{A} = X - \mathcal{J}X,\qquad \forall X \in \mathcal{U}.
	\end{equation}
\end{enumerate}
Запишем уравнение \ref{eq:gamma_x_rule} в матричном виде:
$$
\begin{pmatrix}
		A & 0 \\
		0 & B
\end{pmatrix}
\begin{pmatrix}
		0 & Y_{12} \\
		Y_{21} & 0
\end{pmatrix} -
\begin{pmatrix}
		0 & Y_{12} \\
		Y_{21} & 0
\end{pmatrix}
\begin{pmatrix}
		A & 0 \\
		0 & B
\end{pmatrix} =
\begin{pmatrix}
		0 & X_{12} \\
		X_{21} & 0
\end{pmatrix}	 	
$$
Перемножив и вычтя матрицы получим следующую систему операторный уравнений:
\begin{equation}\label{syst:gamma_x_rule}
	\begin{cases}
		AY_{12} - Y_{12}B = X_{12}, \\
		BY_{21} - Y_{21}A = X_{21},
	\end{cases}
\end{equation}
где $Y_{12}, Y_{21}$ -- искомые операторы. Если оператор $A$ или оператор $B$ ограничен, то уравнения \ref{syst:gamma_x_rule} разрешимы.

Рассмотрим случай, когда $\dim\mathfrak{X}_1 = 1,$ т.\! е. оператор $A$ -- скалярный оператор: $A = \alpha I.$ Перепишем уравнения \ref{syst:gamma_x_rule}:
$$
	\begin{cases}
		\alpha IY_{12} - Y_{12}B = X_{12}, \\
		BY_{21} - Y_{21}\alpha I = X_{21}.
	\end{cases}
$$
Так как оператор $A$ ограничен, то система имеет решение, и оно имеет вид:
$$
	\begin{cases}
		Y_{12} = X_{12}(\alpha I - B)^{-1}, \\
		Y_{21} = (\alpha I - B)^{-1}X_{21}.
	\end{cases}
$$
Таким образом мы получили, что матрица оператора $\Gamma X$ имеет вид:
$$
	\Gamma X = \begin{pmatrix}
		0 & X_{12}(\alpha I - B)^{-1} \\
		(\alpha I - B)^{-1}X_{21} & 0
	\end{pmatrix}.
$$
Оценим норму оператора $\Gamma.$ Пусть $X \in \mathcal{U},$ тогда:
\begin{align*}
&\norm{\Gamma X} \leqslant \max{(\norm{X_{12}(\alpha I - B)^{-1}}, \norm{(\alpha I - B)^{-1}X_{21}})} \leqslant \\ 
&\leqslant \max{(\norm{(\alpha I - B)^{-1}}, \norm{(\alpha I - B)^{-1}})}\norm{X} = \norm{(\alpha I - B)^{-1}}\norm{X} = \gamma \norm{X}.
\end{align*}  
Получили следующую оценку $\norm{\Gamma} \leqslant \gamma.$

\textbf{Теорема~1.}
{ \it Пусть $\lambda_1, \ldots , \lambda_m$ -- собственные значения матрицы $\mathcal{A}$ кратностей $k_1, \ldots , k_m$ соответственно, где $\sum\limits_{i=1}^m k_i = n$. Тогда жорданова форма для матрицы $\mathcal{A}$ имеет вид
$$
	\mathcal{J} = \begin{pmatrix}
		\mathcal{J}_1 & 0 & \dots & 0 \\
		0 & \mathcal{J}_2 & \dots & 0 \\
		\vdots & \vdots & \ddots & \vdots \\
		0 & 0 & \dots & \mathcal{J}_m
	\end{pmatrix}, {\text{где }}
	\mathcal{J}_i = \begin{pmatrix}
		\lambda_i & 1 & 0 & \dots & 0 \\
		0 & \lambda_i & 1 & \dots & 0 \\
		\vdots & \vdots & \vdots & \ddots & \vdots \\
		0 & 0 & 0 & \dots & 1 \\
		0 & 0 & 0 & \dots & \lambda_i
	\end{pmatrix}.
$$
Матрица перехода $\mathcal{U}$ имеет вид
$$
\mathcal{U} = \operatorname{diag} \left( \mathcal{U}_1, \mathcal{U}_2, \dots, \mathcal{U}_m \right),
$$
где матрицы $\mathcal{U}_i \in Matr_{n,k_i} (\mathbb{C}) , i= \overline{1,m}$, имеют вид
$$
\mathcal{U}_i = \begin{pmatrix}
1 & 0 & \dots & 0 \\
\lambda_i & 1 & \dots & 0 \\ 
\lambda_i^2 & 2\lambda_i & \dots & 0 \\
\vdots & \vdots & \ddots & \vdots \\
\lambda_i^{n-1} & (n-1)\lambda_i^{n-2} & \dots & \prod_{k=1}^{k_i-1}(n-k)\lambda_i^{n-k_i}
\end{pmatrix}.
$$
}
\textbf{Доказательство.} Пусть $\lambda_i$ - собственное значение матрицы $\mathcal{A}$ кратности~$k_i$. Найдем соответствующие ему собственный и присоединенные векторы. Рассмотрим матрицу вида
$$
\mathcal{A}-\lambda_i I = 
\begin{pmatrix}
	-\lambda_i & 1 & 0 & \dots & 0 \\
	0 & -\lambda_i & 1 & \dots & 0 \\
	\vdots & \vdots & \vdots & \ddots & \vdots \\
	0 & 0 & 0 & \dots & 1 \\
	\alpha_n & \alpha_{n-1} & \alpha_{n-1} & \dots & \alpha_1-\lambda_i
\end{pmatrix},
$$ 
где $I$ -- единичная матрица.\\
Пусть $x_i \in \mathbb{C}^n$ - собственный вектор, отвечающий собственному значению~$\lambda_i$. Тогда справедливо равенство
\begin{equation}\label{Equation1}
\left(\mathcal{A}-\lambda_i I \right) x_i = 0,
\end{equation}
где $x_i = \left( x_{i1}, x_{i2}, \dots, x_{in} \right)$. Очевидно, что $x_{i1} \ne 0$ (иначе вектор $x_i$ был бы нулевым). Без ограничения общности можно считать, что $x_{i1} = 1$. Тогда равенство \eqref{Equation1} эквивалентно системе уравнений:
\begin{equation}\label{SystemForVector}
\begin{cases}
-\lambda_i + x_{i2} = 0, \\
-\lambda_i x_{i2} + x_{i3} = 0, \\
\dots \\
-\lambda_i x_{i n-2} + x_{i n-1} = 0, \\
\alpha_n + \alpha_{n-1} x_{i2} + \dots + (\alpha_1 - \lambda_i) x_{in} = 0.
\end{cases}
\end{equation}
Решив систему \eqref{SystemForVector}, получим, что 
$$
x_i = \left( 1, \lambda_i, \lambda_i ^2, \dots, \lambda_i ^{n-1} \right).
$$
Пусть $k_i \ne 1$. Найдем присоединенные векторы к вектору $x_i$. \\
Пусть $x_{i,j} \in \mathbb{C}^n,1\le j\le k_i-1$~-~присоединенные векторы, отвечающие собственному значению $\lambda_i$. Первый присоединенный вектор $x_{i,1}$ есть решение уравнения
\begin{equation}\label{Equation2}
\left(\mathcal{A}-\lambda_i I \right) x_{i,1} = x_i,
\end{equation}
где $x_{i,1} =  \left( x_{1,1}, x_{2,1}, \dots, x_{n,1} \right)$ - подлежащий определению присоединенный вектор, $x_i$ - собственный вектор, отвечающий собственному значению $\lambda_i$. Пусть $x_{1,1} = 0$. Тогда равенство \eqref{Equation2} эквивалентно системе уравнений:
\begin{equation}\label{SystemForVect}
\begin{cases}
x_{2,1} = 1, \\
-\lambda_i x_{2,1} + x_{3,1} = \lambda_i, \\
\dots \\
-\lambda_i x_{n-2,1} + x_{n-1,1} = \lambda_i ^{n-2}, \\
\alpha_{n-1} x_{2,1} + \alpha_{n-2} x_{3,1} \dots + (\alpha_1 - \lambda_i ^{n-1}) x_{n,1} = \lambda_i ^{n-1}.
\end{cases}
\end{equation}
Поскольку $P'(\lambda_i)=0$, тогда, решив систему \eqref{SystemForVect}, получим вектор 
$$x_{i,1} = \left(0, 1, 2 \lambda_i, 3 \lambda_i ^2, \dots, (n-1)\lambda_i ^{n-2} \right), $$ который является присоединенным к собственному вектору $x_i$. \\
Аналогичным образом устанавливается, что векторы 
$$
x_{i,p} = \left(\underbrace{0, \dots, 0}_{{p}}, p!, (p+1)! \lambda_i, \dots, \prod_{k=1}^p(n-k)\lambda_i ^{n-p-1} \right),
$$
где $1 \le p \le {k_i-1}$, являются присоединенными к вектору $x_i$. \\
Таким образом, доказано, что матрица $\mathcal{U}$ составлена из собственных и присоединенных векторов. Поэтому $det(\mathcal{U}) \ne 0$, и, следовательно, матрица $\mathcal{U}$ является матрицей перехода к жордановой форме и имеет вид 
$$
\mathcal{U} = \left( \mathcal{U}_1, \mathcal{U}_2, \dots, \mathcal{U}_m \right),
$$
где матрицы $\mathcal{U}_i \in Matr_{n,k_i} (\mathbb{C}), i=1 \dots m$, имеют вид
$$
\mathcal{U}_i = \begin{pmatrix}
1 & 0 & \dots & 0 \\
\lambda_i & 1 & \dots & 0 \\ 
\lambda_i^2 & 2\lambda_i & \dots & 0 \\
\vdots & \vdots & \ddots & \vdots \\
\lambda_i^{n-1} & (n-1)\lambda_i^{n-2} & \dots & \prod_{k=1}^{k_i-1}(n-k)\lambda_i^{n-k_i}
\end{pmatrix}.
$$
Теорема доказана. 

\section{Пример}
Для примера рассмотрим частный случай~$n=3$. Тогда матрица~$\mathcal{A}$ имеет вид:
$$
 \mathcal{A} = 
 \begin{pmatrix}
  0 & 1 & 0\\
  0 & 0 & 1 \\
  \alpha_3 & \alpha_2 & \alpha_1 \\
 \end{pmatrix}
$$
Возможны три варианта:
\begin{enumerate}
\item все собственные значения различны;
\item есть собственные значения $\lambda_1, \lambda_2,$ которые имеют кратности ${k_1=2},~{k_2=1}$ соответственно; 
\item все собственные значения одинаковы;        
\end{enumerate}

Рассмотрим первый случай.
Пусть матрица $\mathcal{A}$ имеет собственные значения $\lambda_1, \lambda_2, \lambda_3$ которые имеют кратности~ $k_1=k_2=k_3=1$ соответственно. Оператор $A$ имеет простую структуру. Тогда жорданова форма матрицы~$\mathcal{A}$ имеет вид:
$$
 \mathcal{J} = 
 \begin{pmatrix}
  \lambda_1 & 0 & 0\\
  0 & \lambda_2 & 0 \\
  0 & 0 & \lambda_3 \\
 \end{pmatrix}
$$
Далее запишем матрицу перехода~$\mathcal{U}$ и~обратную к~ней~$\mathcal{U}^{-1}$:
$$
\begin{aligned}
 \mathcal{U} &= 
 \begin{pmatrix}
  	1 & 1 & 1\\
  	\lambda_1 & \lambda_2 & \lambda_3 \\
  	\lambda_1^2 & \lambda_2^2 & \lambda_3^2 \\
 \end{pmatrix}, \\
 \mathcal{U}^{-1} &= \frac{1}{\Delta} 
 \begin{pmatrix}
  	\lambda_2\lambda_3(\lambda_3-\lambda_2) & \lambda_2^2-\lambda_3^2 & \lambda_3-\lambda_2\\
  	\lambda_1\lambda_3(\lambda_1-\lambda_3) & \lambda_3^2-\lambda_1^2 & \lambda_1-\lambda_3\\
  	\lambda_1\lambda_2(\lambda_2-\lambda_1) & \lambda_1^2-\lambda_2^2 & \lambda_2-\lambda_1\\
 \end{pmatrix},
\end{aligned}
$$
где $\Delta = \det \mathcal{U} = (\lambda_3-\lambda_1)(\lambda_3-\lambda_2)(\lambda_2-\lambda_1).$ \\
Теперь запишем проекторы оператора~$A$:

$$
	\begin{aligned}
		\mathcal{A} &= \mathcal{U}\mathcal{J}\mathcal{U}^{-1}, \\
		\mathcal{P}_i &= \mathcal{U}\mathcal{P}_i'\mathcal{U}^{-1}, \\
		\mathcal{P}_i' &= \left(p_{jk}'\right), \\
		p_{jk}' &= 
		\begin{cases}
			1, & \text{если $j=k=i$;} \\
			0, & \text{в противном случае.}
		\end{cases}
	\end{aligned}
$$
где $i,j,k \in \{1,2,3\}, \text{и}\  \mathcal{P}_i'$~--- проекторы жордановой матрицы~$\mathcal{J}$ 
оператора~$A$. Тогда по теореме о спектральном разложении оператора простой структуры \cite{baskakov}:
$$
	\mathcal{A} = \lambda_1\mathcal{P}_1 + \lambda_2\mathcal{P}_2 + \lambda_3\mathcal{P}_3 
$$


Рассмотрим второй случай.		
Пусть матрица~$\mathcal{A}$ имеет собственные значения~$\lambda_1,~\lambda_2$ 
которые имеют кратности~$k_1=2,~k_2=1$ соответственно. Тогда жорданова форма матрицы~$\mathcal{A}$ имеет вид:
$$
 \mathcal{J} = 
 \begin{pmatrix}
  \lambda_1 & 1 & 0\\
  0 & \lambda_1 & 0 \\
  0 & 0 & \lambda_2 \\
 \end{pmatrix}
$$
Запишем матрицу перехода~$\mathcal{U}$ и~обратную к~ней~$\mathcal{U}^{-1}$:
$$
\begin{aligned}
 \mathcal{U} &= 
 \begin{pmatrix}
  	1 & 0 & 1\\
  	\lambda_1 & 1 & \lambda_2 \\
  	\lambda_1^2 & 2\lambda_1 & \lambda_2^2 \\
 \end{pmatrix}, \\
 \mathcal{U}^{-1} &= \frac{1}{\Delta} 
 \begin{pmatrix}
  	\lambda_2^2-2\lambda_1\lambda_2 & 2\lambda_1 & -1  \\
  	\lambda_1^2\lambda_2 - \lambda_1\lambda_2^2 & \lambda_2^2 - \lambda_1^2 & \lambda_1 - \lambda_2  \\
  	\lambda_1^2 & -2\lambda_1 & 1  \\
 \end{pmatrix},
\end{aligned}
$$
где $\Delta = \det \mathcal{U} = (\lambda_1-\lambda_2)^2$. \\
!!!Дописать про проекторы!!!

И наконец последний случай.		
Пусть матрица~$\mathcal{A}$ имеет одно собственное значение~$\lambda$ 
кратности~$k=3$. Тогда жорданова форма матрицы~$\mathcal{A}$ имеет вид:
$$
 \mathcal{J} = 
 \begin{pmatrix}
  \lambda & 1 & 0\\
  0 & \lambda & 1 \\
  0 & 0 & \lambda \\
 \end{pmatrix}
$$
Запишем матрицу перехода~$\mathcal{U}$ и~обратную к~ней~$\mathcal{U}^{-1}$:
$$
\begin{aligned}
 \mathcal{U} &= 
 \begin{pmatrix}
  	1 & 0 & 0\\
  	\lambda & 1 & 0 \\
  	\lambda^2 & 2\lambda & 2 \\
 \end{pmatrix}, \\
 \mathcal{U}^{-1} &= \frac{1}{\Delta} 
 \begin{pmatrix}
  	1 & 0 & 0  \\
  	-\lambda & 1 & 0  \\
  	\frac{1}{2}\lambda^2 & -\lambda & \frac{1}{2} \\
 \end{pmatrix},
\end{aligned}
$$
где $\Delta = \det \mathcal{U} = 2$. \\
!!!Дописать про проекторы!!!

\vfill

%%%%  ОФОРМЛЕНИЕ СПИСКА ЛИТЕРАТУРЫ %%%
\smallskip \centerline{\bf Литература}\nopagebreak

1. \textit{ Боровских А.В., Перов А.И.} Лекции по обыкновенным дифференциальным уравнениям // Москва-Ижевск: НИЦ  <<Регулярная и хаотическая динамика>>, Институт компьютерных исследований, 2004, 540 стр

2. \textit{ Баскаков А. Г.} Лекции по алгебре // Воронеж: Издательско-полиграфический центр Воронежского государственного университета, 2013, 159 стр

\end{document}
