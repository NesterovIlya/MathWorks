\documentclass{article}
\usepackage[russian]{babel}
\usepackage[utf8]{inputenc}
\usepackage{graphicx}
\usepackage{amsmath}
\usepackage{amsfonts}
\usepackage{amssymb}
\pagestyle{empty} \textwidth=108mm \textheight=165mm
\usepackage{calc,ifthen}

\newcommand{\me}[1]{#1\index{#1}}

\newcounter{first}

\newcommand{\tezis}[5][1]{\vbox{\protect\setcounter{first}{\thepage}%
\begin{center}\textbf{#2}%
\ifthenelse{\not\equal{#3}{}}{\footnotemark[#1]}{}%
\\*\textbf{#4}\\*%
\ifthenelse{\not\equal{#5}{}}{\textit{#5}}{}%
\end{center}\setcounter{equation}{0}%
}\ifthenelse{\not\equal{#3}{}}{\footnotetext[#1]{#3}}{}%
\nopagebreak[4]{}}


\begin{document}

\tezis{ЖОРДАНОВА ФОРМА ДЛЯ ЛОДУ N-ГО ПОРЯДКА}  % Заголовок прописными буквами
{}% Если нет, то поле оставить пустым {}
{\me{ Нестеров И.Н., Клочков С.В., Чурсанова А.С.} (Воронеж)} {nesterovilyan@gmail.com}



%%%%%% Текст тезисов  %%%%%%%

Рассматривается линейное однородное дифференциальное уравнение
$$
	x^{(n)} = \alpha_{1}x^{(n-1)} + \ldots + \alpha_{n}, 
$$
где $\alpha_{k} \in \mathbb{C}, k = \overline{1,n}$. Данное уравнение обычным способом сводится к системе
линейных дифференциальных уравнений вида
$$
	\dot{y} = Ay,
$$
где матрица оператора $A$ имеет вид
$$
	\mathcal{A} = \begin{pmatrix}
		0 & 1 & 0 & \dots & 0 \\
		0 & 0 & 1 & \dots & 0 \\
		\vdots & \vdots & \vdots & \ddots & \vdots \\
		0 & 0 & 0 & \dots & 1 \\
		\alpha_n & \alpha_{n-1} & \alpha_{n-1} & \dots & \alpha_1
	\end{pmatrix},
$$
а $f(\lambda) = \lambda^{n} - \alpha_{1}\lambda^{n-1} - \ldots - \alpha_{n}, \lambda \in \mathbb{C}$ -- характеристический многочлен этой матрицы.

\textbf{Теорема~1.}
{ \it Пусть $\lambda_1, \ldots , \lambda_m$ собственные значения матрицы $\mathcal{A}$ кратностей $k_1, \ldots , k_m$ соответственно. Тогда жорданова форма для матрицы $\mathcal{A}$ имеет вид:
$$
	\mathcal{J} = \begin{pmatrix}
		\mathcal{J}_1 & 0 & \dots & 0 \\
		0 & \mathcal{J}_2 & \dots & 0 \\
		\vdots & \vdots & \ddots & \vdots \\
		0 & 0 & \dots & \mathcal{J}_m
	\end{pmatrix}, {\text{где }}
	\mathcal{J}_i = \begin{pmatrix}
		\lambda_i & 1 & 0 & \dots & 0 \\
		0 & \lambda_i & 1 & \dots & 0 \\
		\vdots & \vdots & \vdots & \ddots & \vdots \\
		0 & 0 & 0 & \dots & 1 \\
		0 & 0 & 0 & \dots & \lambda_i
	\end{pmatrix}.
$$
Векторами столбцами матрицы перехода являются собственные и присоединенные векторы, где собственные векторы имеют вид: 
$$x^i =(1, \lambda_i, \ldots, \lambda_i^{n-1}), 1 \leqslant i \leqslant m,$$
а присоединенные:
$$y_i^j = (1^{(j)}, \lambda_i^{(j)}, \ldots , (\lambda_i^{n-1})^{(j)}), 1 \leqslant i \leqslant m, 1 \leqslant j \leqslant k_i.$$} 


%%%%  ОФОРМЛЕНИЕ СПИСКА ЛИТЕРАТУРЫ %%%
\smallskip \centerline{\bf Литература}\nopagebreak

1. \textit{Львовский С. М.} Набор и верстка в системе LATEX. М.:
МЦНМО, 2003. — 448 с.

2. \textit{ Ульянов П. Л.} О классах бесконечно дифференцируемых
функций // Матем. сб. - 1990. - Т. 181,  № 5. -  С. 589-609.

\end{document}
