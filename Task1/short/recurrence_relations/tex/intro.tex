Рассмотрим задачу построения последовательности $x : \mathbb{N} \rightarrow \mathbb{C}$, для которой известны первые $n \ge 1$ членов $x(1) = x_1^0, \dots, x(n) = x_n^0$ и все последующие задаются реккурентными соотношениями
\begin{equation}\label{Relation}
x(m+n) = \alpha_1 x(m+n-1) + \alpha_2 x(m+n-2) + \dots + \alpha_n x(m), m \ge 1,
\end{equation}
где $\alpha_1, \dots, \alpha_n$ - известные числа из $\mathbb{C}$.
Введем в рассмотрение последовательность $x_k : \mathbb{N} \rightarrow \mathbb{C}, 1 \le k \le n$, определенные равенствами
\begin{align*}
x_1(m) &= x(m), \\
x_2(m) &= x(m+1), \\
\dots \\
x_n(m) &= x(m+n), n \ge 1.
\end{align*}
Используя \eqref{Relation}, получаем, что такие последовательности связаны друг с другом равенствами
\begin{equation}\label{Equality}
\begin{aligned}
x_1(m+1) &= x_2(m), \\
x_2(m+1) &= x_3(m), \\
\dots \\
x_n(m+1) &= \alpha_n x_1(m) + \alpha_{n-1} x_2(m)+ \dots + \alpha_1 x_n(m), m \ge 1.
\end{aligned}
\end{equation}
Рассмотрим последовательность $y: \mathbb{Z} \rightarrow \mathbb{C}^n$ вида
$$
y(m) = (x_1(m), \dots, x_n(m)), m \ge 1.
$$
Из \eqref{Equality} следует, что для нее верны равенства 
\begin{equation}\label{Sequence}
y(m+1) = A y(m), m \ge 1,
\end{equation}
где оператор $A: \mathbb{C}^n \to \mathbb{C}^n$ определен с помощью матрицы 
$$
	\mathcal{A} = \begin{pmatrix}
		0 & 1 & 0 & \dots & 0 \\
		0 & 0 & 1 & \dots & 0 \\
		\vdots & \vdots & \vdots & \ddots & \vdots \\
		0 & 0 & 0 & \dots & 1 \\
		\alpha_n & \alpha_{n-1} & \alpha_{n-1} & \dots & \alpha_1
	\end{pmatrix},
$$
а $P(\lambda) = \lambda^{n} - \alpha_{1}\lambda^{n-1} - \ldots - \alpha_{n}, \lambda \in \mathbb{C}$ -- характеристический многочлен этой матрицы.