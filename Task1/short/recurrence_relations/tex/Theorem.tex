\textbf{Теорема~1.}
{ \it Пусть $\lambda_1, \ldots , \lambda_m$ -- собственные значения матрицы $\mathcal{A}$ кратностей $k_1, \ldots , k_m$ соответственно, где $\sum\limits_{i=1}^m k_i = n$. Тогда жорданова форма для матрицы $\mathcal{A}$ имеет вид
$$
	\mathcal{J} = \begin{pmatrix}
		\mathcal{J}_1 & 0 & \dots & 0 \\
		0 & \mathcal{J}_2 & \dots & 0 \\
		\vdots & \vdots & \ddots & \vdots \\
		0 & 0 & \dots & \mathcal{J}_m
	\end{pmatrix}, {\text{где }}
	\mathcal{J}_i = \begin{pmatrix}
		\lambda_i & 1 & 0 & \dots & 0 \\
		0 & \lambda_i & 1 & \dots & 0 \\
		\vdots & \vdots & \vdots & \ddots & \vdots \\
		0 & 0 & 0 & \dots & 1 \\
		0 & 0 & 0 & \dots & \lambda_i
	\end{pmatrix}.
$$
Матрица перехода $\mathcal{U}$ имеет вид
$$
\mathcal{U} = \operatorname{diag} \left( \mathcal{U}_1, \mathcal{U}_2, \dots, \mathcal{U}_m \right),
$$
где матрицы $\mathcal{U}_i \in Matr_{n,k_i} (\mathbb{C}) , i= \overline{1,m}$, имеют вид
$$
\mathcal{U}_i = \begin{pmatrix}
1 & 0 & \dots & 0 \\
\lambda_i & 1 & \dots & 0 \\ 
\lambda_i^2 & 2\lambda_i & \dots & 0 \\
\vdots & \vdots & \ddots & \vdots \\
\lambda_i^{n-1} & (n-1)\lambda_i^{n-2} & \dots & \prod_{k=1}^{k_i-1}(n-k)\lambda_i^{n-k_i}
\end{pmatrix}.
$$
}
\textbf{Доказательство.} Пусть $\lambda_i$ - собственное значение матрицы $\mathcal{A}$ кратности~$k_i$. Найдем соответствующие ему собственный и присоединенные векторы. Рассмотрим матрицу вида
$$
\mathcal{A}-\lambda_i I = 
\begin{pmatrix}
	-\lambda_i & 1 & 0 & \dots & 0 \\
	0 & -\lambda_i & 1 & \dots & 0 \\
	\vdots & \vdots & \vdots & \ddots & \vdots \\
	0 & 0 & 0 & \dots & 1 \\
	\alpha_n & \alpha_{n-1} & \alpha_{n-1} & \dots & \alpha_1-\lambda_i
\end{pmatrix},
$$ 
где $I$ -- единичная матрица.\\
Пусть $x_i \in \mathbb{C}^n$ - собственный вектор, отвечающий собственному значению~$\lambda_i$. Тогда справедливо равенство
\begin{equation}\label{Equation1}
\left(\mathcal{A}-\lambda_i I \right) x_i = 0,
\end{equation}
где $x_i = \left( x_{i1}, x_{i2}, \dots, x_{in} \right)$. Очевидно, что $x_{i1} \ne 0$ (иначе вектор $x_i$ был бы нулевым). Без ограничения общности можно считать, что $x_{i1} = 1$. Тогда равенство \eqref{Equation1} эквивалентно системе уравнений:
\begin{equation}\label{SystemForVector}
\begin{cases}
-\lambda_i + x_{i2} = 0, \\
-\lambda_i x_{i2} + x_{i3} = 0, \\
\dots \\
-\lambda_i x_{i n-2} + x_{i n-1} = 0, \\
\alpha_n + \alpha_{n-1} x_{i2} + \dots + (\alpha_1 - \lambda_i) x_{in} = 0.
\end{cases}
\end{equation}
Решив систему \eqref{SystemForVector}, получим, что 
$$
x_i = \left( 1, \lambda_i, \lambda_i ^2, \dots, \lambda_i ^{n-1} \right).
$$
Пусть $k_i \ne 1$. Найдем присоединенные векторы к вектору $x_i$. \\
Пусть $x_{i,j} \in \mathbb{C}^n,1\le j\le k_i-1$~-~присоединенные векторы, отвечающие собственному значению $\lambda_i$. Первый присоединенный вектор $x_{i,1}$ есть решение уравнения
\begin{equation}\label{Equation2}
\left(\mathcal{A}-\lambda_i I \right) x_{i,1} = x_i,
\end{equation}
где $x_{i,1} =  \left( x_{1,1}, x_{2,1}, \dots, x_{n,1} \right)$ - подлежащий определению присоединенный вектор, $x_i$ - собственный вектор, отвечающий собственному значению $\lambda_i$. Пусть $x_{1,1} = 0$. Тогда равенство \eqref{Equation2} эквивалентно системе уравнений:
\begin{equation}\label{SystemForVect}
\begin{cases}
x_{2,1} = 1, \\
-\lambda_i x_{2,1} + x_{3,1} = \lambda_i, \\
\dots \\
-\lambda_i x_{n-2,1} + x_{n-1,1} = \lambda_i ^{n-2}, \\
\alpha_{n-1} x_{2,1} + \alpha_{n-2} x_{3,1} \dots + (\alpha_1 - \lambda_i ^{n-1}) x_{n,1} = \lambda_i ^{n-1}.
\end{cases}
\end{equation}
Поскольку $P'(\lambda_i)=0$, тогда, решив систему \eqref{SystemForVect}, получим вектор 
$$x_{i,1} = \left(0, 1, 2 \lambda_i, 3 \lambda_i ^2, \dots, (n-1)\lambda_i ^{n-2} \right), $$ который является присоединенным к собственному вектору $x_i$. \\
Аналогичным образом устанавливается, что векторы 
$$
x_{i,p} = \left(\underbrace{0, \dots, 0}_{{p}}, p!, (p+1)! \lambda_i, \dots, \prod_{k=1}^p(n-k)\lambda_i ^{n-p-1} \right),
$$
где $1 \le p \le {k_i-1}$, являются присоединенными к вектору $x_i$. \\
Таким образом, доказано, что матрица $\mathcal{U}$ составлена из собственных и присоединенных векторов. Поэтому $det(\mathcal{U}) \ne 0$, и, следовательно, матрица $\mathcal{U}$ является матрицей перехода к жордановой форме и имеет вид 
$$
\mathcal{U} = \left( \mathcal{U}_1, \mathcal{U}_2, \dots, \mathcal{U}_m \right),
$$
где матрицы $\mathcal{U}_i \in Matr_{n,k_i} (\mathbb{C}), i=1 \dots m$, имеют вид
$$
\mathcal{U}_i = \begin{pmatrix}
1 & 0 & \dots & 0 \\
\lambda_i & 1 & \dots & 0 \\ 
\lambda_i^2 & 2\lambda_i & \dots & 0 \\
\vdots & \vdots & \ddots & \vdots \\
\lambda_i^{n-1} & (n-1)\lambda_i^{n-2} & \dots & \prod_{k=1}^{k_i-1}(n-k)\lambda_i^{n-k_i}
\end{pmatrix}.
$$
Теорема доказана. 